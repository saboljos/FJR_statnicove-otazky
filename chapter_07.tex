\section{Prinicipy metody Monte-Carlo}

\subsection{Základní filosofie}

Základním principem pro deterministické kódy je přímé řešení matematických rovnic popisujících chování v jaderném reaktoru, tedy difúzní rovnice, nebo transportní rovnice. Máme tedy jasný matematický model a ten řešeíme pomocí numeriky.\\

Oproti tomu stochastické metody ke svému řešení žádné rovnice nepotřebují a simulují přímo konkrétní částice. Modelování tedy probíhá na takovém prinicpu, že se jasně stanoví geometrie a pomocí generátoru náhodných čísel se generují imaginární částice, které s danými geometriemi interagují. To, jak budou částice interagovat čistě závisí na účinných průřezech (je potřeba znát diferenciální průřezy, jelikož nás i zajímá směr příletu/odletu částice). \\

Z této základní filosofie tedy platí jasné rozdíly mezi deterministickými a stochastickými metodami:\\

Deterministické:
\begin{itemize}
  \item Rychlejší, ale přesné pouze tak, jak přesný je daný matematický model.
  \item Získám jeden výsledek bez nejistot (ačkoliv nejistoty jsou dané zjednodušeními, není je tak jednoduché vyčíslit).
  \item Složité modelování geometrie $\rightarrow$ aplikují se různá geometrická zjednodušení (nekonečná mříž, 2D aproximace apod.)
  \item Účinné průřezy se musejí grupovat $\rightarrow$ je potřeba více programů a více přístupů (nejprve příprava v rámci jednoho souboru pomocí transportky, poté aplikace ve velkém celozónovém modelu pomocí difúzní rovnice).
\end{itemize}

Stochastické:
\begin{itemize}
  \item Pomalejší, ale teoreticky přesnější. S rozvojem techniky se dá očekávat převládnutí. Přesnost je ovlivněna hlavně účinnými průřezy, a pak trochu generací pseudonáhodných čísel.
  \item Vždy mi nástroj vyplivne výsledek s danou nejistotou, kterou je teoreticky možné snížit simulací většího počtu částic.
  \item Jednoduché 3D modelování.
  \item Spojité účinné průřezy (lze získat pomocí NJOY).
\end{itemize}

Stochatické metody tedy nepotřebují žádné matematické rovnice, celý výpočet probíhá pomocí tzv. \textbf{náhodné procházky} metodou Monte-Carlo, což je dále rozebráno v další otázce.\\

Jako částice je možné simulovat absolutně cokoliv, záleží na znalosti účinných průřezů a rozšíření výpočetního kódu:
\begin{itemize}
  \item Serpent 2 -- hlavní využití pro reaktorovou fyziku, specializuje se tedy převážně na neutrony (ačkoliv by měl nějak zohledňovat i (n,$\gamma$) reakce, ze zkušenosti se mi zdá, že to moc nefunguje).
  \item MCNP 6 -- momentálně asi nejznámnější kód, umí modelovat neutrony, fotony, elektrony..., skoro cokoliv. Bere se jako benchmark pro ostatní kódy, ale je složitý na syntaxi a zdlouhavý.
  \item OpenMC -- o tom moc nevím. Dobrý je, že je zadarmo.
  \item KENO -- low-budget alternativa od Scaleu. Prý stojí za prd, ale umí hezky spolupracovat s ostatními SCALE balíky, tudíž se taky může hodit.
\end{itemize}

\subsection{Analogová a neanalogová metoda}

Základním principem \textbf{analogové metody} je to, že trekujeme konkrétní neutrony, které když zaniknou (absorbcí či únikem), tak prostě zaniknou a dál neexistují. Při výpočtu skórování v datém detektoru/tallies se tedy započítávají s váhou 1. Tato metoda více odpovídá realitě, nicméně je zdlouhavá, jelikož pro přesnější výsledek je nutné simulovat větší populaci částic.\\

To platí např. pro prostředí s velkou absorbcí a malým zdrojem (chceme-li např. řešit problematiku stínění, reflektoru apod.)\\

Oproti tomu \textbf{neanalogová metoda} či \textbf{implicitní metoda} funguje tak, že narodí-li se částice, tak je jí přiřazena váha 1. A tím, jak se částice toulá po daném systému, tak interaguje s prostředím až do místa, kde by teoreticky mělo dojít k její absorbci. Nicméně, nedojde k úplné absorbci, pouze se dojde k započítání do skórování a dále se jí srazí váha na nižší hodnotu. Nedojde tedy k zániku, a jedna částice může pokračovat v toulání se po systému. Pouze má nižší váhu, proto narazí-li na další absorbci, opět dojde k započtení do skórování, ale s nižší váhou, jelikož v tomto druhém místě by už částice neměla teoreticky existovat. Dojde-li po několika srážkách ke snížení váhy pod jistou hladinu, až tehdy dojde k celkovému zániku dané částice.\\

Tímto způsobem se dají zpřesnit výsledky za pomoci menšího počtu částic. Nicméně, jde o komplikovanější metodu a je potřeba znát širší matematický základ. V prvním případě je důležité určit, jakým zpsůobem se změní váha dané částice. Tato technika se nazývá \textbf{redukce variance} (redukce variance, jelikož zpřesňují výsledek a snižuji rozptyl, tedy varianci).

\subsection{Redukce variance}

První možností při sražení váhy je za pomoci teoretické absorbce. To se určí jednoduše dle vztahu:

$$ w = w' \left ( 1-\dfrac{\Sigma_\text{a}}{\Sigma_\text{t}} \right ). $$

Dále je třeba určit, při jaké energii dojde ke skutečnému zániku částice, jelikož trekování částice s příliš malou váhou je naopak kontraproduktivní. To se určuje dle tzv. Ruské rulety, jejíž nejjednodušší zápis je ve tvaru:

$$ P = 1-\dfrac{w}{w_0}, $$

kde $w_0$ značí váhu na začátku historie a $P$ značí pravděpodobnost, při které dojde k zániku.

Dále je možné podobnou metodu aplikovat na vznik částic (z hlediska neutronů štěpení (n,xn) reakce). V takovém případě se nová váha určí jako:

$$ w = w' \nu. $$

Tohle jsou jenom základní metody, ve skutečnosti je nutné aplikovat více metod redukce variance. Těmi hlavními jsou:

\begin{itemize}
  \item cut-off -- zánik při poklesu pod jistou hladinu (energetické, váhové, časové apod.),
  \item splitting -- rozdělení na 2 částice, každá s jinou váhou (geometrické, energetické, časové),
  \item ruská ruleta -- naopak zánik dané částice, kombinace se splittingem (velká váha splitting, malá váha ruská ruleta),
  \item biassing -- to je to zmiňované sražení váhy (absorbcí).
\end{itemize}

Nicméně nejdůležitější je pochopitelně kombinace všeho.
