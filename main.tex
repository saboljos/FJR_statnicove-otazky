\documentclass[a4paper, 11pt]{article}

\usepackage[version=3]{mhchem}
\usepackage{siunitx}
\usepackage{graphicx}
\usepackage{amsmath}
\usepackage[total={16cm,25cm}, top=3cm, left=2.5cm, includefoot]{geometry}
\usepackage{pgfplots}
\usepackage{float}
\usepackage{amsfonts}
\usepackage{booktabs}
\usepackage{easytable}
\usepackage{mathtools}
\usepackage{comment}
\usepackage[makeroom]{cancel}
\renewcommand{\figurename}{Obrázek}
\renewcommand{\refname}{Zdroje}
\renewcommand{\tablename}{Tabulka}
\renewcommand{\contentsname}{Obsah}

\setlength\parindent{0pt}
\setcounter{section}{0}
\renewcommand{\labelenumi}{\alph{enumi}.}
\usepackage{xcolor}
\definecolor{light-gray}{gray}{0.95}
\newcommand{\code}[1]{\colorbox{light-gray}{\texttt{#1}}}
\newcommand{\logoCVUT}{\includegraphics[width = 0.5\textwidth]{img/symbol_cvut_konturova_verze_cb.pdf}}
\newcommand{\logoFJFI}{\includegraphics[width = 0.4\textwidth]{img/fjfi_logo.png}}

\begin{document}
% \include{veličiny}
% \include{zkratky}

% titulní strana
\thispagestyle{empty}

\begin{center}
	{\LARGE
		České vysoké učení technické v Praze \par
		Fakulta jaderná a fyzikálně inženýrská
	}
    \vspace{10mm}

    \begin{tabular}{c}
		\textbf{Katedra jaderných reaktorů} \\[3pt]
    \end{tabular}

   \vspace{10mm} \logoCVUT \vspace{15mm}

   {\huge \textbf{Fyzika jaderných reaktorů}\par}
   \vspace{5mm}
   {\huge \textbf{Magisterské studium}\par}

   \vspace{15mm}
   {\Large \MakeUppercase{Státnicové otázky}}

   \vfill
   {\large
    \begin{tabular}{ll}
    Rok: & 2025
    \end{tabular}
   }
\end{center}

% Prohlášení
\newpage
\thispagestyle{empty}

%~
\vfill

\vspace{1em}
Čau ahoj. Právě čtete soubor magisterských státnicových otázek k předmětu Fyzika jaderných reaktorů, který byl vypracován za dlouhých zimních večerů v pokojíku ve Švýcarsku. Jedná se o sepis všech prezentací, skript a materiálů, ze kterých jsme během výuky čerpali, a které jsou dle mého názoru k pochopění nejdůležitější. Šlo zejména o:

\begin{itemize}
    \item prezentace k předmětu ZAF (J. Frýbort, L. Frýbortová, M. Štefánik),
    \item prezentace k předmětu FARE (J. Frýbort),
    \item prezentace k předmětu DERF (J. Frýbort a P. Suk),
    \item přednášky z předmětu SMRF (O. Huml),
    \item přednášky z předmětu KID (O. Huml),
    \item zápisky z předmětu KID (O. Huml),
    \item Dynamika jaderných reaktorů -- B. Heřmanský,
    \item Nuclear Reactor Physics -- W. Stacey,
    \item Introduction to Nuclear Engineering -- J. Lamarsch,
    \item Development of a New Monte Carlo Reactor Physics Code -- J. Leppänen,
    \item manuál ke kódu NJOY,
    \item manuál ke kódu SCALE,
    \item manuál ke kódu MCNP,
    \item manuál ke kódu Serpent,
    \item prezentace MCNP z LANL,
    \item a spoustu dalšího.
\end{itemize}

Pokud najdete chybu, hoďte issue na Git: \it{https://github.com/saboljos/FJR\_statnicove-otazky}.\\

\rm Hodně zdaru!

\rm J.S.

\vspace{2em}

\clearpage{\pagestyle{empty}}

\include{none}

\newpage
\parskip=0pt
\begin{small}
\tableofcontents
\end{small}
\parskip=7pt
\newpage

\section[Difúzní teorie]{Odvození a využití difuzní teorie v reaktorové fyzice}

\textbf{Difúzní rovnice} představuje základní rovnici stanovující prostorové rozložení neutronů v látce. Přesný popis neutronů je příliš komplikovaný, proto je vhodné volit určitá zjednodušení. Pro nejobecnější popis je nutné použít Boltzmannovu transportní rovnici, nicméně s určitými zanedbáními postačuje pouze difúuzní rovnice, případně difúzní rovnice s transportními korekčními faktory.

\subsection{Fickův zákon}

Difúzní rovnice je založena na \textbf{Fickově zákonu}, který je možné znát např. z chemie (šíření látky proti směru gradientu koncentrace):

\begin{equation}
    \boxed{ \textbf{J} = - D \text{grad} \Phi = -D \nabla \phi, }
\end{equation}

kde $\textbf{J}$ značí hustotu proudu neutronů (vektor), $D$ značí difúzni koeficient a $\Phi$ značí hustotu toku neutronů. To samé z části platí i pro neutrony. Pokud je hustota toku neutronů v jedné oblasti vyšší, neutrony putují opačným směrem. 

To a jak jsou neutrony ochotny se v dané látce šířít pak popisuje \textbf{difúzní koeficient} $D$. Difúzní rovnice předpokládá, že je difúzní koeficient funkcí materiálu, nikoliv polohy (izotropní rozptyl). V případě slabého anizotropního rozptylu je možné difúzní koeficient zavést za pomoci \textbf{střední volné dráhy pro transport} $\lambda_\text{tr}$ jako:

$$ D = \dfrac{\lambda_\text{tr}}{3}, $$

$$ \lambda_\text{tr} = \dfrac{1}{\Sigma_\text{tr}} = \dfrac{1}{\Sigma_\text{s}(1-\bar{\mu})}, $$

kde střední volná dráha pro transport udává průměrnou vzdálenost, kterou neutron urazí v původním směru po nekonečném množství rozptylových srážek. Pro izotropní rozptyl v těžišťové soustavě pak navíc platí:

$$ \bar{\mu} = \dfrac{2}{3A}. $$

$\bar{\mu}$ zde označuje cosinus úhlu rozptylu.

Obecně Fickův zákon platí všude, až na:

\begin{itemize}
    \item silně absorbční prostředí (absorbátory),
    \item vzdálenost menší než 3 střední volné dráhy pro transport od zdroje, nebo od hranic difúzního prostředí,
    \item silně anizotropní rozptyl.
\end{itemize}

Tyto podmínky jsou částečně všude, proto je potřeba k nim přistupovat opatrně, případně hledat jiná přiblížení.

\subsection{Odvození}

Odvození vychází z rovnice kontinuity. Pokud si představíme element objemu $V$, pak musí platit:

\begin{equation*}
    \boxed{
    \begin{pmatrix} \text{rychlost změny} \\ \text{počtu neutronů} \end{pmatrix} = \begin{pmatrix} \text{rychlost produkce} \\ \text{neutronů} \end{pmatrix} - \begin{pmatrix} \text{rychlost absorbce} \\ \text{neutronů} \end{pmatrix} - \begin{pmatrix} \text{rychlost úniku} \\ \text{(difuze) neutronů} \end{pmatrix}.
    }
\end{equation*}

Pokud $n(\textbf{r})$ označuje hustotu neutronů, pak \textbf{rychlost změny počtu neutronů} lze určit jako:

$$ \dfrac{\text{d}}{\text{d}t} \int_V n(\textbf{r}) \: \text{d} V, $$

přičemž derivaci je možné z nějaké chytré matematické věty šoupnout do integrálu, tedy:

$$ \boxed{\int_V \dfrac{\partial n}{\partial t} \: \text{d} V.} $$

\textbf{Rychlost produkce}, resp. \textbf{rychlost absorbce} je možné zapsat za pomoci reakčních rychlostí:

$$ \boxed{\int_V s(\textbf{r}) + \nu \Sigma_\text{f} \Phi(\textbf{r}) \: \text{d} V,} $$

$$ \boxed{\int_V \Sigma_\text{a} \Phi(\textbf{r}) \: \text{d} V.} $$

Rychlost úniku z elementu $V$ je pak dána difúzí a aplikací Fickova zákonu přes plochu $S$:

$$ \oint_S \textbf{J} \cdot \text{d} \textbf{S}, $$

což po aplikaci další chytré matematické věty pojmenované po nějakém slavném matematikovi přejde na:

$$ \boxed{\int_V \text{div} \textbf{J} \: \text{d} V.} $$

Teď už se to jenom poskládá dohromady, odstraní integrál (předpoklad, že to platí všude, tudíž se z integrální formy přejde do diferenciální), z divergence gradientu se udělá Laplacián (předpoklad, že $D$ není funkcí prostoru, a tudíž půjde vytkntout) a ještě se od hustoty neutronů přejde k hustotě toku neutronů, čímž dostaneme:

\begin{equation}
    \boxed{
        \dfrac{\partial n}{\partial t} = \dfrac{1}{v} \dfrac{\partial \Phi}{\partial t} = D \Delta \Phi - \Sigma_\text{a} \Phi + s + \nu \Sigma_\text{f} \Phi.
    }
\end{equation}

Ještě je možný jeden zápis, kdy difúuzní rovnici podělím difúzním koeficientem, čímž se do rovnice dostane difúzní délka $L$, resp. difúzní plocha $L^2$:

$$ L^2 = \dfrac{D}{\Sigma_\text{a}}, $$

\begin{equation}
    \boxed{
        \dfrac{1}{D v} \dfrac{\partial \Phi}{\partial t} = \Delta \Phi - \dfrac{1}{L^2} \Phi + \dfrac{s}{D} + \dfrac{\nu \Sigma_\text{f}}{D} \Phi.
    }
\end{equation}

\textbf{Difúzní plocha} popisuje látku vzhledem k tomu, kolik je neutron schopný procestovat v prostředí před tím, než bude absorbován. Pokud se bude uvažovat pravděpodobnost absorbce, která se zintegruje přes trasu a nalezne se střední hodnota tak vyjde, že $L^2$ představuje 1/6 střední hodnoty čtverce přímé zvdálenosti mezi místem vzniku a místem absorbce.

Hodnoty $L$ a $D$ se liší od materiálu, pro základní moderátory platí:

\begin{table}[h!]
    \centering
    \begin{tabular}{lccc}
    \toprule
    \textbf{Moderátor} & \textbf{hustota (g/cm$^3$)} & $D$ \textbf{(cm)} & $L$ \textbf{(cm)} \\ \midrule
    H\textsubscript{2}O & 1,00 & 0,13 & 2,62 \\
    D\textsubscript{2}O & 1,10 & 0,76 & 147 \\
    Be & 1,85 & 0,45 & 22 \\
    Grafit & 1,60 & 0,87 & 61 \\ \bottomrule
    \end{tabular}
    \caption{Tabulka vlastností moderátorů.}
\end{table}
    

\subsection{Řešení}

Jedná se o diferenciální rovnici, tudíž pro její řešení je nutné stanovit okrajové podmínky. Mezi ty patří:

\begin{itemize}
    \item Konečnost a nezápornost hustoty toku neutronů $0 \leq \Phi < \infty $.
    \item Rovnost hustoty toku a proudu na rozhraní $\Phi_A = \Phi_B$ \& $\textbf{J}_A = \textbf{J}_B$.
    \item Okrajová podmínka na vnějším rozhraní.
    \item Zdrojové podmínky.
\end{itemize}

Podmínku na vnějším rozhraní je možné zavést za pomoci \textbf{extrapolované délky} $d$, jelikož bylo zjištěno, že pokud hustota toku klesne na nulu ve vzdálenosti $d$ od rozhraní, pak se spočtené rozložení blíží realitě. Extrapolovanou délku je možné stanovit jako:

$$ d = 0,71 \lambda_\text{tr} = 2,13 D. $$

Zdrojové podmínky se týkají geometrie \textbf{zdroje} $s$. Jelikož difúzní rovnice neplatí v místě zdroje, je nezbytné tyto oblasti řešit samostatně. Postupuje se tak, že se difúzka řeší mimo zdroj, přičemž zdroj samotný se obklopí plochou, kterou prochází všechny emitované neutrony o vydatnosti $S$. Limitní podmínku je pak možné stanovit dle geometrie jako:

\begin{itemize}
    \item $ \lim_{x \to 0} 2 \: \textbf{J}(x) = S $ pro rovinný zdroj (2 plochy),
    \item $ \lim_{r \to 0} 4 \pi r^2 \: \textbf{J}(r) = S $ pro bodový zdroj (plocha sféry),
    \item $ \lim_{r \to 0} 2 \pi r \: \textbf{J}(r) = S $ pro přímkový zdroj (obvod kružnice).
\end{itemize}

Difúzka je jednoduše řešitelná i analyticky, kor stacionární případ. Stačí nalézt vhodného Laplaciána dle geometrie a přepsat rovnici do tvaru:

$$ \Delta \Phi \pm A^2 = 0 $$

Výsledné řešení je pak závislé na znamínku. Pokud převládá absorbce nad produkcí (ZAF1, difúze v prostředí $ - A^2$), vede řešení na hyperbolické funkce a Modifikované Besselovy funkce typu $I_n$ a $K_n$. Pokud převládá produkce nad absorbcí (ZAF2, homogenní reaktor $ + A^2$), vede řešení na goniometrické funkce a Besselovy funkce typu $J_n$ a $Y_n$.

\subsection{Grupová difúzní rovnice}

Difúzní rovnice je možné zapsat i s energetickou závislotí za pomoci rozgrupování. Poté se pro každou grupu řeší vlastní difúzka, přičemž přechod mezi grupami se zapisuje pomocí \textbf{rozptylového grupového průžezu} $\Sigma_{m \to n}$, tedy pravděpodobnost přechodu z grupy $m$ do grupy $n$. Grupa s nejvyšší energií se označuje indexem $1$ a s klesající energií roste číslo grupy. Skupinová difúzní rovnice pro grupu $g$ pak vypadá (stacionární případ, kde zdroj ze štěpení je zahrnut v koeficientu $s$):

\begin{equation}
    \boxed{
        D_g \Delta \Phi_g - \Sigma_\text{a,g} \Phi_g - \sum^N_{h=g+1} \Sigma_{g \to h} \Phi_g + \sum^{g-1}_{h=1} \Sigma_{h \to g} \Phi_h = -s_g.
    }
\end{equation}

Zde předpokládáme pouze downscattering, tedy přesun z vyšší grup do nižší grupy (resp. z nižšího pořadového čísla do vyššího). To obecně platí pro vysoké energie (neutrony vznikají s energií v řádech MeVů a postupně termalizují, tedy pouze ztrácejí energii rozptylovými srážkami), na úrovní tepelné energie se může objevovat i upscattering a celkový popis je složitější.

\subsection{2G přiblížení \& grupování}

Obecně je nutné uvažovat alespoň 2 grupy, tepelné a rychlé neutrony s hranicí na 0,625 eV. Ukazuje se, že hustotu toku tepelných neutronů je možné popsat za pomoci \textbf{Maxwellovy distribuce}. Pak je možné účinné průřezy vystředovat a uvažovat celou tepelnou oblast jako jednu grupu se středovanými průřezy. 

Maxwellova distribuce je dána:

$$ n(E) = \dfrac{2 \pi n}{(\pi k T)^{3/2}} \sqrt{E} \: \text{exp} \left(- \dfrac{E}{kT}\right), $$

rychlost je definuje:

$$ v(E) = \sqrt{\dfrac{2E}{m}}, $$

a pak je možné určit hustotu toku tepelných neutronů $\phi_T$ jako:

$$ \Phi_T = \int_T \Phi(E) \: \text{d}E = \int_T n(E) v(E) \: \text{d}E = \dfrac{2n}{\sqrt{\pi}} \sqrt{\dfrac{2kT}{m}} $$

Tím se určí střední hustota toku tepelných neutronů, která se pak aplikuje na středované průřezy:

$$ \bar{\Sigma} = \dfrac{1}{\Phi_T} \int_T \Sigma(E) \Phi(E) \: \text{d}E.$$

Pokud navíc platí oblast $1/v$ (a pokud ne, přidá se korekční faktor $g$), tak je možné nahradit:

$$ \Sigma(E) \Phi(E) = \Sigma(E_0) \Phi_0, $$

kde index 0 odpovídá energii 0,0253 eV, tedy 293 K, což se nalezne v tabulkách. Celkové výrazy pak budou:

$$ \boxed{\bar{\Sigma} = \dfrac{\sqrt{\pi}}{2} g(T) \Sigma(E_0) \sqrt{\dfrac{T_0}{T}},} $$

což platí pro všechny hodnoty (průřezy, energie, difúzní koeficienty apod.).

Nyní je už možné zkompletovat 2G sadu difúzních rovnic:

\begin{equation}
    \boxed{
        \Delta \Phi_T - \dfrac{1}{L_T^2} \Phi_T = -\dfrac{\Sigma_1 \Phi_1}{\bar{D}},
    }
\end{equation}

\begin{equation}
    \boxed{
        \Delta \Phi_1 - \dfrac{1}{\tau} \Phi_1 = -\dfrac{s_1}{D_1},
    }
\end{equation}

kde $\tau$ označuje Fermiho stáří neutronů (1/6 čtverce přímé vzdálenosti mezi vznikem a přechodem do tepelné grupy, příp. absorbce), a kde zdrojem tepelných neutronů je scattering z rychlé grupy, a zdroje v rychlé grupě jsou označeny jako $s_1$ (např. ze štěpení).

\subsection{Využití}

Difúzka se využívá jako první přiblížení, pokud nás zajímá transport neutronů, rozložení neutronového pole, kritičnost reaktoru, efekt moderátoru a reflektoru apod. Využívá se i v praxi, jelikož její řešení je velmi jednoduché. Např. pokud řešíme celozónový výpočet, tak většina všech full-core nástrojů využívají právě difuzní řešení (Andrea, PARCS apod.), které se řeší numericky (nodálně, sítěmi apod.). Z lattice kódu se vytáhne sada účinných průřezů a v full-core kódu se tyhle průřezy nastrkají do difúzní rovnice.

Mezi rovnice odvozené v rámci ZAF patří:

\subsubsection{Modifikovaná 2G kritická rovnice}

Vychází z předpokladu, že máme homogenní systém popsaný za pomoci 2G rovnice. Neutrony vznikají ze štěpení jako rychlé neutrony, přecházejí do tepelné oblasti kde inicializují štěpení. Celá úloha je pak zestaciarizována za pomoci koeficientu násobení. Ke štěpení rychlými neutrony nedochází vůbec.

$$ k_\text{ef} = \dfrac{k_\infty}{1 + B^2 \cdot M^2}, $$

kde $k_\infty$ představuje koeficient násobení pro systém bez úniku, $B^2$ představuje geometrický faktor (dáno geometrií) a $M^2$ představuje migrační plochu (dáno materiálem).

Na rovnici je možné pohlížet i jako na pravděpodobnost, že neutron neunikne $P_\text{NL}$. Pokud se navíc migrační plocha přepíše jako suma difúzní plochy (charakterizuje transport tepelných neutronů, tedy proces difuze) a Fermiho stáří (charakterizuje transport rychlých neutronů, tedy proces termalizace), pak platí:

$$ P_\text{NL} = \dfrac{k_\text{ef}}{k_\infty} = \dfrac{1}{1 + B^2 \cdot (L^2 + \tau)} \approx \dfrac{1}{1 + B^2 \cdot L^2} \cdot \dfrac{1}{1 + B^2 \cdot \tau} = P_\text{TNL} \cdot P_\text{FNL}. $$

Rovnici je možné i zobecnit pro konkrétnější případy, více grup apod.
\section[Metody jader a vlastních funkcí]{Zavedení metody jader a metody vlastních funkcí pro řešení úloh z reaktorové fyziky a transportu záření}
\section[Transportní rovnice]{Odvození a využití transportní teorie v reaktorové fyzice}

Transportní rovnice je základní matematický model používaný v reaktorové fyzice k popisu pohybu a interakce neutronů v materiálu. Její cílem je určit rozložení neutronů v prostoru, energii a čase v jaderném reaktoru nebo jiném prostředí.

Momentálně se jedná o rovnici, která co nejvěrněji popisuje chování a šíření neutronů, nicméně její zápis je příliš složitý a neexistuje obecné analytické řešení. Z toho důvodu je nutné přecháet k určitým zjednodušením, kterými může být například difúzní rovnice.

\subsection{Boltzmanova integro-diferenciální transportní rovnice}

Odvození musí nutně vycházet z dějů, které má popisovat. Vzhledem k běžným podmínkám v reaktorech je možné popis chování neutronů zúžit na úlohu neutrálních částic pohybujících se podle zákonitostí klasické mechaniky s kvantovým popisem jejich interakcí s jádry okolního prostředí.

Cílem je převést izolovanou částicovou povahu interakcí neutronů na spojitou veličinu.

\subsubsection{Předpoklady}

Pro odvození musíme předpokládat:

\begin{itemize}
  \item dostatečná populace neutronů, abychom mohli uvažovat statistické středování,
  \item neutrony interagují pouze s jádry okolí (která jsou v klidu) a nikoliv mezi sebou $\rightarrow$ v reaktorech splněno vždy (projevuje se až pro $\Phi > 10^{20}$ 1/cm$^2$/s).
\end{itemize}

\subsubsection{Matematický aparát}

Pro odvození je potřeba zavést:

\textbf{a) Nezávislé proměnné}:

Nezávislé proměnné představují souřadnice, pomocí kterých jsme schopni neutronům přiřadit bod fázového prostoru. Pro jednoznačné určení je potřeba 6~nezávislých proměnných (v klasické mechanice jde o 3~složky polohového vektoru a 3~složky vektoru rychlosti), nicméně v tomto případě se od vektoru rychlosti přechází ke směru pohybu a velikosti rychlosti.

Pro popis tedy předpokládáme:

\begin{itemize}
  \item polohový vektor $\textbf{r} = (x, y, z)$ -- 3 složky,
  \item směrový vektor $\Omega = (\phi, \theta)$ -- 2 složky,
  \item (kinetická) energie $E$ -- 1 složka.
\end{itemize}

Pro případný přepočet poté platí:

$$ v = \sqrt{\dfrac{2m}{E}}, $$
$$ \textbf{v} = v \Omega. $$

Časovou závislost zanedbáváme a na čas nahlížíme jako na parametr, nikoliv souřadnici.

\textbf{b) Fázový prostor:}

Představuje prostor pro popis neutronových interakcí. Základním předpokladem je zachování částicové povahy neutronů při využití statistické povahy pohybu velkého množství částic. Jednotlivé srážky se proto odehrávají ve statisticky středovaném \textbf{elementu fázového prostoru} $\Delta \textbf{P}$:

$$ \Delta \textbf{P} = \Delta \textbf{r} \Delta \Omega \Delta \textbf{E}. $$

Počet neutronů v $\Delta \textbf{P}$ se může změnit vlivem změny v dílčích parametrech, tedy:

\begin{itemize}
  \item v $\Delta \textbf{r}$ -- únikem z $\Delta \textbf{r}$, případně absorbcí nebo vznikem v $\Delta \textbf{r}$,
  \item v $\Delta \Omega$ -- rozptylem,
  \item v $\Delta E$ -- zpomalením, nebo zrychlením.
\end{itemize}

\textbf{c) Závislé proměnné:}

Dále je potřeba zavést závislé proměnné (závislé od toho, že jsou závislé na těch nezávislých a času), které představují sledované veličiny.

Sem řadíme 3 základní:

\begin{itemize}
  \item úhlová hustota neutronů $n(\textbf{r}, \Omega, E, t)$ (1/cm$^3$) -- skalár,
  \item úhlová hustota toku neutronů $\Omega(\textbf{r}, \Omega, E, t)$ (1/cm$^2$/s) -- skalár (ale bacha, neplést se směrovým vektorem $\Omega$),
  \item úhlová hustota proudu neutronů $\textbf{J}(\textbf{r}, \Omega, E, t)$ (1/cm$^2$/s) -- vektor
\end{itemize}

s definovanými vztahy:

\begin{equation}
  \boxed{
  \Phi(\textbf{r}, \Omega, E, t) = v \cdot n(\textbf{r}, \Omega, E, t),
  \label{definice_hustota_toku}}
\end{equation}

\begin{equation}
  \boxed{
  \textbf{J}(\textbf{r}, \Omega, E, t) = \Omega \cdot \Phi(\textbf{r}, \Omega, E, t).
  \label{definice_hustota_proudu}}
\end{equation}

Pro kontext, \textbf{$\Phi$ udává celkový součet drah všech neutronů za sekundu v jednotkovém objemu fázového prostoru}, zatímco \textbf{$\textbf{J}$ udává počet neutronů, který prochází na steradián, energii a plochu ve směru $\Omega$ plochou kolmou k $\Omega$ v čase}. Tedy, zatímco $n$ a $\Phi$ se vztahují k objemu, $\textbf{J}$ se vztahuje k ploše.

\subsubsection{Odvození}

Nejprve je potřeba si uvědomit, jak popisujeme interakce neutronů s jádry. Pravděpodobnost interakce je dána $\sigma$ (cm$^2$) a $\Sigma$ (1/cm), přičemž mezi možné interakce řadíme pouze: rozptyl, radiační záchyt a štěpení, případně absorbci (což je součet štěpení a radiačního záchytu).

\textbf{Makroskopický účinný průřez pro interakci $i$ na jádře $j$} -- $\Sigma_{ij}(\textbf{r}, E, t)$ definujeme jako podíl pravděpodobnosti interakce neutronu reakcí $i$ s jádrem $j$ na jednotku délky dráhy.

Dále \textbf{reakční rychlost pro interakci $i$ na jádře $j$} -- $F_{ij}(\textbf{r}, E, t)$ určíme jako:

$$ F_{ij}(\textbf{r}, E, t) = \Sigma_{ij}(\textbf{r}, E, t) \Phi(\textbf{r}, E, t). $$

Při předpokladu nezávislosti jednotlivých interakcí (abychom mohli průřezy sčítat) je možné definovat totální makroskopický účinný průřez:

$$ \Sigma_i(\textbf{r}, E, t) = \sum_{j=1}^J \Sigma_{ij}(\textbf{r}, E, t) = \sum_{j=1}^J N_j(\textbf{r}, t) \sigma_{ij}(E). $$

Dále si pro popis rozptylu zavedeme \textbf{diferenciální rozptylové jádro} -- $f_s(\Omega' \cdot \Omega, E' \rightarrow E)$ (-) tak, že:

\begin{itemize}
  \item $f_s(\Omega' \cdot \Omega, E' \rightarrow E) \Delta \Omega \Delta E$ značí poměrnou pravděpodobnost rozptylu ze směru $\Omega'$ a energie $E'$ do rozsahu směrů $\Delta \Omega$ a energií $\Delta E$,
  \item pro rozptyl na jádře $j$ platí:
\end{itemize}
$$\Sigma_{sj}(\textbf{r}, \Omega' \cdot \Omega, E' \rightarrow E, t) = \Sigma_{sj}(\textbf{r}, E', t) f_s(\Omega' \cdot \Omega, E' \rightarrow E), $$

\begin{itemize}
  \item pro účely normalizace platí:
\end{itemize}
$$\int_\mathbb{R^+} \int_{4 \pi} f_s(\Omega' \cdot \Omega, E' \rightarrow E) \text{d} \Omega \text{d}E = 1. $$


Pro odvození musíme vycházet z bilanční rovnice neutronů, kde sledujeme změnu počtu neutronů v libovolném elementu fázového prostoru tvořeném objemem $V$ a $\Delta \Omega \Delta E$ během časového intervalu $\Delta t$.

Jednodušeji řečeno posuzujeme jednotlivé příspěky či ztráty neutronů:

\begin{equation}
  \boxed{
  \begin{pmatrix} \text{počet neutronů} \\ \text{v čase } t + \Delta t \end{pmatrix} = \begin{pmatrix} \text{počet neutronů} \\ \text{v čase } t \end{pmatrix} + \begin{pmatrix} \text{zisk neutronů} \\ \text{během } \Delta t \end{pmatrix} - \begin{pmatrix} \text{ztráta neutronů} \\ \text{během } \Delta t \end{pmatrix}.
  \label{bilancni_rovnice}}
\end{equation}

Nyní popíšeme jednotlivé stavy:

\textbf{a) Počet neutronů v čase $t + \Delta t$:}

$$ \Delta \Omega \Delta E \int_V n(\textbf{r}, \Omega, E, t + \Delta t) \text{d}V $$

\textbf{b) Počet neutronů v čase $t$:}

$$ \Delta \Omega \Delta E \int_V n(\textbf{r}, \Omega, E, t) \text{d}V $$

\textbf{c) Zisk neutronů během $\Delta t$:}

Zisk neutronů se skládá ze zisku:

\begin{itemize}
  \item rozptylem:
\end{itemize}
$$ \Delta \Omega \Delta E \Delta t \int_V \int_\mathbb{R^+} \int_{4 \pi} f_s(\Omega' \cdot \Omega, E' \rightarrow E) \Sigma_s(\textbf{r}, E', t) \Phi(\textbf{r}, \Omega', E', t) \text{d}\Omega' \text{d}E' \text{d}V, $$

\begin{itemize}
  \item štěpením:
\end{itemize}
$$ \Delta \Omega \Delta E \Delta t \dfrac{\chi(E)}{4 \pi} \int_V \int_\mathbb{R^+} \int_{4 \pi} \nu(E') \Sigma_f(\textbf{r}, E', t) \Phi(\textbf{r}, \Omega', E', t) \text{d}\Omega' \text{d}E' \text{d}V. $$


Štěpení se předpokládá izotropní (skutečně je) a neuvažují se zpožděné neutrony. Pro distribuční funkci $\chi(E)$ musí platit:

$$ \int_\mathbb{R^+} \chi(E) \text{d} E = 1. $$

\textbf{d) Ztráta neutronů během $\Delta t$:}

Ztráta neutronů se skládá ze ztráty:

\begin{itemize}
  \item rozptylem:
\end{itemize}
$$ \Delta \Omega \Delta E \Delta t \int_V \int_\mathbb{R^+} \int_{4 \pi} f_s(\Omega' \cdot \Omega, E \rightarrow E') \Sigma_s(\textbf{r}, E, t) \Phi(\textbf{r}, \Omega, E, t) \text{d}\Omega' \text{d}E' \text{d}V. $$

Veličiny $\Sigma_s$ a $\Phi$ nejsou závislé na čárkovaných veličinách (u zisku neutronů jsou, tam to provést nejde), takže je možné je z integrálu $\text{d}\Omega'$ a $\text{d}E'$ vytknout a využít normalizace rozptylového jádra (viz někde nahoře), čímž dostaneme:

$$ \Delta \Omega \Delta E \Delta t \int_V  \Sigma_s(\textbf{r}, E, t) \Phi(\textbf{r}, \Omega, E, t) \text{d}V, $$

\begin{itemize}
  \item absorbcí:
\end{itemize}
$$ \Delta \Omega \Delta E \Delta t \int_V  \Sigma_a(\textbf{r}, E, t) \Phi(\textbf{r}, \Omega, E, t) \text{d}V, $$

\begin{itemize}
  \item únikem:
\end{itemize}
$$ \Delta \Omega \Delta E \Delta t \oint_S  \textbf{J}(\textbf{r}, \Omega, E, t) \cdot \text{d}\textbf{S} = \Delta \Omega \Delta E \Delta t \int_V  \text{div} \textbf{J}(\textbf{r}, \Omega, E, t) \text{d}V.$$

Dohromady to vše dává nepřehlednou motanici:

\small
\begin{equation*}
\begin{multlined}
  \Delta \Omega \Delta E \int_V n(\textbf{r}, \Omega, E, t + \Delta t) \text{d}V = \Delta \Omega \Delta E \int_V n(\textbf{r}, \Omega, E, t) \text{d}V + \\
  + \Delta \Omega \Delta E \Delta t \int_V \int_\mathbb{R^+} \int_{4 \pi} f_s(\Omega' \cdot \Omega, E' \rightarrow E) \Sigma_s(\textbf{r}, E', t) \Phi(\textbf{r}, \Omega', E', t) \text{d}\Omega' \text{d}E' \text{d}V + \\
  + \Delta \Omega \Delta E \Delta t \dfrac{\chi(E)}{4 \pi} \int_V \int_\mathbb{R^+} \int_{4 \pi} \nu(E') \Sigma_f(\textbf{r}, E', t) \Phi(\textbf{r}, \Omega', E', t) \text{d}\Omega' \text{d}E' \text{d}V - \\ 
  - \Delta \Omega \Delta E \Delta t \int_V  \Sigma_s(\textbf{r}, E, t) \Phi(\textbf{r}, \Omega, E, t) \text{d}V - \Delta \Omega \Delta E \Delta t \int_V  \Sigma_a(\textbf{r}, E, t) \Phi(\textbf{r}, \Omega, E, t) \text{d}V - \\
  - \Delta \Omega \Delta E \Delta t \int_V  \text{div} \textbf{J}(\textbf{r}, \Omega, E, t) \text{d}V
\end{multlined}
\end{equation*}
\normalsize

Vykrátíme rovnici diferencemi $\Delta \Omega \Delta E \Delta t$, z definice nahradíme $n = \dfrac{\Phi}{v}$ a $\textbf{J} = \Omega \cdot \Phi$, uzavřeme do jednoho integrálu přes objem, výrazy přesuneme na jednu stranu a trochu přeskládáme:

\small
\begin{equation*}
\begin{multlined}
  \int_V \left [ \dfrac{1}{v} \left [ \dfrac{\Phi(\textbf{r}, \Omega, E, t + \Delta t) - (\textbf{r}, \Omega, E, t)}{\Delta t} \right ] + \left [ \Sigma_a + \Sigma_s \right ] \cdot \Phi(\textbf{r}, \Omega, E, t) + \text{div} \left [ \Omega(\textbf{r}, \Omega, E, t) \cdot \Phi(\textbf{r}, \Omega, E, t) \right ] \right ] \text{d}V - \\
  - \int_V \int_\mathbb{R^+} \int_{4 \pi} \left [ f_s(\Omega' \cdot \Omega, E' \rightarrow E) \Sigma_s(\textbf{r}, E', t) \Phi(\textbf{r}, \Omega', E', t) - \dfrac{\chi(E)}{4 \pi} \nu(E') \Sigma_f(\textbf{r}, E', t) \Phi(\textbf{r}, \Omega', E', t) \right ] \text{d}\Omega' \text{d}E' \text{d}V = 0.
\end{multlined}
\end{equation*}
\normalsize

Dále se limitně přejde z diferencí k diferenciálům, čímž se první zlomek s $\Phi$ změní v parciální derivaci. Součet průřezů dá: $\Sigma_s + \Sigma_a = \Sigma_t$, z rozepsání divergence přes indexy a aplikace derivace součinu vznikne: $\text{div} \left [ \Omega \cdot \Phi(\textbf{r}, \Omega, E, t) \right ] = \Omega \cdot \text{grad} \Phi(\textbf{r}, \Omega, E, t)$, součin rozptylového jádra s průřezem pro rozptyl dá: $f_s(\Omega' \cdot \Omega, E' \rightarrow E) \Sigma_s(\textbf{r}, E', t) = \Sigma_s(\textbf{r}, \Omega' \cdot \Omega, E' \rightarrow E, t)$ a zbavíme se integrálu přes objem, čímž lze dospět k tvaru:

\small
\begin{equation*}
\begin{multlined}
  \left [ \dfrac{1}{v} \dfrac{\partial}{\partial t} + \Sigma_t(\textbf{r}, E, t) + \Omega \cdot \text{grad} \right ]\Phi(\textbf{r}, \Omega, E, t) = \\
  = \int_\mathbb{R^+} \int_{4 \pi} \left [ \Sigma_s(\textbf{r}, \Omega' \cdot \Omega, E' \rightarrow E, t) + \dfrac{\chi(E)}{4 \pi} \nu(E') \Sigma_f(\textbf{r}, E', t)\right ] \Phi(\textbf{r}, \Omega', E', t) \text{d}\Omega' \text{d}E'.
\end{multlined}
\end{equation*}
\normalsize

Ještě se na pravou stranu přidá zdrojová podmínka $Q(\textbf{r}, \Omega, E, t)$, čímž se dochází k \textbf{Boltzmanově integro-diferenciální transportní rovnici}:

\begin{equation}
  \boxed{
  \begin{multlined}
    \left [ \dfrac{1}{v} \dfrac{\partial}{\partial t} + \Sigma_t(\textbf{r}, E, t) + \Omega \cdot \text{grad} \right ]\Phi(\textbf{r}, \Omega, E, t) = \\
    = \int_\mathbb{R^+} \int_{4 \pi} \left [ \Sigma_s(\textbf{r}, \Omega' \cdot \Omega, E' \rightarrow E, t) + \dfrac{\chi(E)}{4 \pi} \nu(E') \Sigma_f(\textbf{r}, E', t)\right ] \Phi(\textbf{r}, \Omega', E', t) \text{d}\Omega' \text{d}E' + \\
    + Q(\textbf{r}, \Omega, E, t).
  \end{multlined}}
  \label{integro-diferencialni_transportka}
\end{equation}

\subsubsection{Podmínky platnosti a řešitelnost}

Pro zopakování je důležité znát podmínky platnosti takovéto rovnice. Jde o předpoklady, které se musely zavést pro odvození:

\begin{itemize}
  \item Srážky je možné popisovat zvlášť od pohybu částic, tj. procesy se nijak neovlivňují a na každý je možné zavést speciální fyziální aparát (klasická mechanika pohybu vs. kvantový popis interakcí).
  \item Časové trvání srážek je zanedbatelné k času mezi srážkami.
  \item Srážky mezi neutrony se zanedbávají.
  \item Dostatečná populace neutronů umožňující středování přes element fázového prostoru.
\end{itemize}

V reaktorové fyzice je možné teorii bez obavu použít, jelikož populace se pohybuje cca v rozmezí 10$^{15}$ 1/cm$^2$/s, což je vhodný kompromis mezi dostatečnou populací pro středování a limitní populací, nad kterou už neutrony interagují mezi sebou.

Jelikož se jedná o integro-diferenciální rovnici, je třeba znát okrajové a počáteční podmínky. Nicméně zatím neexistuje numerické, natož analytické řešení v libovolné 3D geometrii $\rightarrow$ jsou potřeba různé zjednodušení:

\begin{itemize}
  \item stacionární případ,
  \item směrové omezení.
\end{itemize}

Takovýmito zjednodušeními je např. možné přejít zpět k difúzní teorii. Pro numerické řešení je dále třeba dostatečné množství jaderných dat (průřezy, štěpné spektrum apod.), aplikuje se bodové řešení:

\begin{itemize}
  \item Rozsekání energetické spojitosti na grupy,
  \item rozsekání směrové spojitosti pomocí S$_n$ metody,
  \item rozsekání prostoru pomocí sítí a diferencí.
\end{itemize}

Víc je o tom v další kapitole.

\subsection{Integrální tvar transportní rovnice}

Jedná se o případ, kdy se aplikuje integrování a středování Boltzmanovy stacionární rovnice přes dráhu neutronu $\textbf{s}$ ve směru $\Omega$ mezi body $\textbf{r}$ a $\textbf{r}'$. 

\subsubsection{Odvození}

Je to trochu čarování. Předpokládáme tedy, že se neutron pohybuje po dráze $\textbf{s}$ ve směru $\Omega$ mezi body $\textbf{r}$ a $\textbf{r}'$, což ve vektorovém zápisu je možné zapsat jako:

$$\textbf{r}'(s) = \textbf{r} + s \cdot \Omega. $$
  
Vyjde se z výrazu:

$$\Phi(\textbf{r}', \Omega, E) \cdot \text{exp} \left ( -\int_0^s \Sigma_t(\textbf{r}'(\tilde{s}), E) \text{d} \tilde{s} \right ),$$
  
což vlastně představuje vystředování $\Phi$ přes dráhu $\textbf{s}$. Výraz se zderivuje podle $\textbf{s}$, trošku se přeskládají závorky, dosadí se Boltzmanova transportní rovnice ve stacionárním tvaru a nakonec se výraz zpětně zintegruje přes $\textbf{s}$ od 0 do nekonečna. Ještě se v odvození vyhodí člen se štěpením a zahrne se do zdrojové podmínky (z nepřehledného výrazu se stane míň nepřehledný). Nechce se mi to vypisovat, stejně se to nebude nikdo učit nazpaměť, ale není to nic složitého. Nakonec vyjde \textbf{Integrální tvar transportní rovnice}:

\begin{equation}
  \boxed{
  \begin{multlined}
    \Phi(\textbf{r}, \Omega, E) = \int_0^\infty \text{exp} \left ( -\int_0^s \Sigma_t(\textbf{r} + \Omega \tilde{s}, E) \text{d} \tilde{s} \right ) \\
    \left [ \int_\mathbb{R^+} \int_{4 \pi}  \Sigma_s(\textbf{r} + \Omega s, \tilde{\Omega} \cdot \Omega, \tilde{E} \rightarrow E) + \Sigma_s(\textbf{r} + \Omega s, \tilde{\Omega}, \tilde{E}) \text{d}\tilde{\Omega} \text{d}\text{E} + Q(\textbf{r} + \Omega s, \Omega, E) \right ] \text{d}s.
  \end{multlined}}
  \label{integralni_transportka}
\end{equation}

V ofiko prezentacích na FARE jsou v závorkách mínusy, protože je jinak zavedená vektorová notace. Ale dle mého je to logičtější (a i správnější) takhle. Tak jak je obrázek v prezentaci, tak se nejde od $\textbf{r}$ do $\textbf{r}'$, ale opačně, což pak nedává smysl a je tam chyba. Nebo si pamatujte výraz z prezentace, ale pak ignorujte ten obrázek, který tomu neodpovídá.

\subsection{Možná řešení}

\subsubsection{Deska}

Desková geometrie (1D) a monoenergetické přiblížení (1G) ve stacionárním stavu:

$$\left [ \mu \dfrac{\partial}{\partial x} + \Sigma \right ] \Phi(x, \mu) = Q(x, \mu),$$

\noindent kde $\mu$ představuje směrovou závislost a platí $\mu = \text{cos}(\vartheta)$. Předpokládají se konstantní průřezy, deska v rozmezí $0$ až $a$ a že zdroje neutronů jsou umístěny pouze v desce. Poté jsou okrajové podmínky ve tvaru:

$$\Phi(0, \mu) = 0, \mu > 0$$
$$\Phi(a, \mu) = 0, \mu < 0$$

Výsledkem jsou funkce:

$$\Phi(x, \mu) = \dfrac{1}{\mu} \int_0^x e^{\dfrac{-\Sigma (x-x')}{\mu}} Q(x', \mu) \text{d}x', \mu > 0$$
$$\Phi(x, \mu) = \dfrac{1}{\mu} \int_a^x e^{\dfrac{-\Sigma (x-x')}{\mu}} Q(x', \mu) \text{d}x', \mu > 0$$

Tyto integrály není možné analyticky vyjádřit v obecném tvaru pro $Q(x, \mu)$, ale je to možné například pro konstantní člen $Q_0$. Pokud bychom se chtěli zbavit i směrové závislosti, musel by se výraz $\Phi(x, \mu)$ vyintegrovat přes $\mu$ (180° rozmezí pro $\vartheta$, tedy od -1 do +1 pro $\mu$), čímž by se z diferenciální hustoty toku stala "normální" hustota toku. Není to vidět, ale obě veličiny jsou rozdílné, mají i rozdílné jednotky (1/cm$^2$/s/rad vs. 1/cm$^2$/s). Takže bacha jaké píšete proměnné, je potřeba si to ohlídat.

\subsection{Využití}

DOPSAT!!!
\section[$S_n$ a $P_n$ metody]{Základy $S_n$ a $P_n$ metody a metody charakteristik, diskretizace proměnných v transportní rovnici a rozvoj úhlových závislostí do Legendrových polynomů}

Momentálně neexistuje obecné analytické řešení transportní rovnice. Pokud ji chceme řešit, je nutné přejít ke zjednodušení. Obvykle se aplikuje přechod ze spojitosti na bodovou aproximaci, tj. grupové rozsekání pro energii, metoda sítí pro prostor a S$_n$ metoda pro směr. Výsledkem je strašně moc elementů a do té doby, než se začnou prodávat kvantové počítače, tak pro celozónový výpočet nevhodné.

Dále se pro zjednodušení aplikuje oddělení velikosti a úhlové závislosti průřezu pro rozptyl, protože je závislý na všech proměnných a složitě by se vyčísloval. Zní to sice nóbl, ale ve výsledku se z funkce o více proměnných udělá součin dvou funkcí o méně proměnných pomocí rozvoje do Legendrových polynomů  (tím se oddělí směrová závislost) a sférických harmonických funkcí (tím se oddělí závislost původního a nového směru). 

\subsection{Metody řešení transportní rovnice}

Z teorie Fourierova rozkladu platí:

$$\Sigma_s(\textbf{r}, \Omega' \cdot \Omega, E' \rightarrow E) = \sum_{l=0}^{\infty} \dfrac{2l + 1}{4 \pi} \Sigma_{s,l}(\textbf{r}, E' \rightarrow E) P_l(\Omega' \cdot \Omega),$$
$$\Sigma_{s,l}(\textbf{r}, E' \rightarrow E) = \int_{-1}^1 \Sigma_s(\textbf{r}, \Omega' \cdot \Omega, E' \rightarrow E) P_l(\Omega' \cdot \Omega) \text{d}\mu,$$

\noindent kde $\Sigma_{s,l}$ je $l$-tý Legendrův koeficient pro rozptyl a $P_l$ je $l$-tý Legandrův polynom. Tím se oddělil prostor od směru. Dále se Legendrův polynom rozloží na sférické harmonické funkce:

$$P_l(\Omega' \cdot \Omega) = \sum_{m=-l}^{+l} R_l^m(\Omega) R_l^m(\Omega'),$$

\noindent kde $R_l^m$ jsou zmiňované sférické harmonické funkce. Tím se oddělil původní a nový směr. Rozptyl pak vypadá:

$$\Sigma_s(\textbf{r}, \Omega' \cdot \Omega, E' \rightarrow E) = \sum_{l=0}^{\infty} \sum_{m=-l}^{+l} \dfrac{2l + 1}{4 \pi} \Sigma_{s,l}(\textbf{r}, E' \rightarrow E) R_l^m(\Omega) R_l^m(\Omega').$$

Matematická vsuvka: Legandrovy polynomy i sférické harmonické funkce jsou bázové funkce, jsou tedy navzájem ortonormální (Legendrovy polynomy nejsou ortogonální, proto je tam ten normalizační zlomek). Legendrovy polynomy rozkládají prostor $(-1, +1)$ se skalárním součinem definovaným přes obyčejný integrál a sférické harmonické funkce sféru $(0, 4 \pi)$ a je možné je získat jako vlastní funkce Laplaciánu.

Dále se zavádějí \textbf{úhlové momenty} $\phi_l^m$, které zjednodušují zápis. Jde o průmět do sférických harmonických funkcí, takže zjednodušeně řečeno, pokud jsou sférické harmonické funkce bází prostoru, tak vezmu hustotu toku a pomocí superpozice ji vyjádřím v těchto bázových souřadnicích:

$$\phi_l^m(\textbf{r}, E') = \int_{4 \pi} R_l^m (\Omega') \Phi(\textbf{r}, \Omega', E') \text{d}\Omega'.$$

Pro následný zápis transportní rovnice pak tedy platí:

\begin{equation*}
  \begin{multlined}
    \int_\mathbb{R^+} \int_{4 \pi} \Sigma_s(\textbf{r}, \Omega' \cdot \Omega, E' \rightarrow E) \Phi(\textbf{r}, \Omega', E') \text{d}\Omega' \text{d}E' = \\
    = \int_\mathbb{R^+} \int_{4 \pi} \left [ \sum_{l=0}^{\infty} \sum_{m=-l}^{+l} \dfrac{2l + 1}{4 \pi} \Sigma_{s,l}(\textbf{r}, E' \rightarrow E) R_l^m(\Omega) R_l^m(\Omega') \right ] \Phi(\textbf{r}, \Omega', E') \text{d}\Omega' \text{d}E' = \\
    = \sum_{l=0}^{\infty} \dfrac{2l + 1}{4 \pi} \sum_{m=-l}^{+l} R_l^m(\Omega) \int_\mathbb{R^+} \Sigma_{s,l}(\textbf{r}, E' \rightarrow E) \left ( \int_{4 \pi} R_l^m(\Omega') \Phi(\textbf{r}, \Omega', E') \text{d}\Omega' \right ) \text{d}E' =\\
    = \sum_{l=0}^{\infty} \dfrac{2l + 1}{4 \pi} \sum_{m=-l}^{+l} R_l^m(\Omega) \int_\mathbb{R^+} \Sigma_{s,l}(\textbf{r}, E' \rightarrow E) \phi_l^m(\textbf{r}, E') \text{d}E'
  \end{multlined}
\end{equation*}

A kvůli tomu se to dělá. Je to hezká matematika, ale reálně tomu na katedře skutečně rozumí jenom pár lidí, takže je asi zbytečný to umět nazpaměť.

\subsubsection{S$_n$ metoda}

Jedná se o směrovou diskretizaci prostorového úhlu $\Omega$. Neznámé poté představují úhlové hustoty toku neutronů $\Phi_d(\textbf{r})$ v $n$ vybraných směrech, čímž se z transportní rovnice odstraní směrová závislost. Pro výsledné řešní platí:

$$\Phi(\textbf{r}, \Omega) \approx \sum_{d=1}^{n} \Phi_d(\textbf{r}) \delta(1-\Omega \cdot \Omega_d).$$

\subsubsection{P$_n$ metoda}

Vyjde se z rozvoje do Legendrových polynomů a sférických harmonickcý funkcí, který se utne ve stupni $n$. Tento rozvoj se poté zpětně dosadí do transportní rovnice (viz předešlé kompletní dosazení) a řeší se soustava $(n+1)^2$ rovnic, ve kterých neznámou hraje úhlový moment $\phi_l^m(\textbf{r})$. Tím se úplně zbavíme závislosti na $\Omega$. Výsledné řešení bude tvaru:

$$\Phi(\textbf{r}, \Omega) \approx \sum_{l=0}^{\infty} \dfrac{2l + 1}{4 \pi} \sum_{m=-l}^{+l} R_l^m(\Omega)  \phi_l^m(\textbf{r}). $$
\include{kapitoly/chapter_05}
\include{kapitoly/chapter_06}
\section{Prinicipy metody Monte-Carlo}

\subsection{Základní filosofie}

Základním principem pro deterministické kódy je přímé řešení matematických rovnic popisujících chování v jaderném reaktoru, tedy difúzní rovnice, nebo transportní rovnice. Máme tedy jasný matematický model a ten řešeíme pomocí numeriky.\\

Oproti tomu stochastické metody ke svému řešení žádné rovnice nepotřebují a simulují přímo konkrétní částice. Modelování tedy probíhá na takovém prinicpu, že se jasně stanoví geometrie a pomocí generátoru náhodných čísel se generují imaginární částice, které s danými geometriemi interagují. To, jak budou částice interagovat čistě závisí na účinných průřezech (je potřeba znát diferenciální průřezy, jelikož nás i zajímá směr příletu/odletu částice). \\

Z této základní filosofie tedy platí jasné rozdíly mezi deterministickými a stochastickými metodami:\\

Deterministické:
\begin{itemize}
  \item Rychlejší, ale přesné pouze tak, jak přesný je daný matematický model.
  \item Získám jeden výsledek bez nejistot (ačkoliv nejistoty jsou dané zjednodušeními, není je tak jednoduché vyčíslit).
  \item Složité modelování geometrie $\rightarrow$ aplikují se různá geometrická zjednodušení (nekonečná mříž, 2D aproximace apod.)
  \item Účinné průřezy se musejí grupovat $\rightarrow$ je potřeba více programů a více přístupů (nejprve příprava v rámci jednoho souboru pomocí transportky, poté aplikace ve velkém celozónovém modelu pomocí difúzní rovnice).
\end{itemize}

Stochastické:
\begin{itemize}
  \item Pomalejší, ale teoreticky přesnější. S rozvojem techniky se dá očekávat převládnutí. Přesnost je ovlivněna hlavně účinnými průřezy, a pak trochu generací pseudonáhodných čísel.
  \item Vždy mi nástroj vyplivne výsledek s danou nejistotou, kterou je teoreticky možné snížit simulací většího počtu částic.
  \item Jednoduché 3D modelování.
  \item Spojité účinné průřezy (lze získat pomocí NJOY).
\end{itemize}

Stochatické metody tedy nepotřebují žádné matematické rovnice, celý výpočet probíhá pomocí tzv. \textbf{náhodné procházky} metodou Monte-Carlo, což je dále rozebráno v další otázce.\\

Jako částice je možné simulovat absolutně cokoliv, záleží na znalosti účinných průřezů a rozšíření výpočetního kódu:
\begin{itemize}
  \item Serpent 2 -- hlavní využití pro reaktorovou fyziku, specializuje se tedy převážně na neutrony (ačkoliv by měl nějak zohledňovat i (n,$\gamma$) reakce, ze zkušenosti se mi zdá, že to moc nefunguje).
  \item MCNP 6 -- momentálně asi nejznámnější kód, umí modelovat neutrony, fotony, elektrony..., skoro cokoliv. Bere se jako benchmark pro ostatní kódy, ale je složitý na syntaxi a zdlouhavý.
  \item OpenMC -- o tom moc nevím. Dobrý je, že je zadarmo.
  \item KENO -- low-budget alternativa od Scaleu. Prý stojí za prd, ale umí hezky spolupracovat s ostatními SCALE balíky, tudíž se taky může hodit.
\end{itemize}

\subsection{Analogová a neanalogová metoda}

Základním principem \textbf{analogové metody} je to, že trekujeme konkrétní neutrony, které když zaniknou (absorbcí či únikem), tak prostě zaniknou a dál neexistují. Při výpočtu skórování v datém detektoru/tallies se tedy započítávají s váhou 1. Tato metoda více odpovídá realitě, nicméně je zdlouhavá, jelikož pro přesnější výsledek je nutné simulovat větší populaci částic.\\

To platí např. pro prostředí s velkou absorbcí a malým zdrojem (chceme-li např. řešit problematiku stínění, reflektoru apod.)\\

Oproti tomu \textbf{neanalogová metoda} či \textbf{implicitní metoda} funguje tak, že narodí-li se částice, tak je jí přiřazena váha 1. A tím, jak se částice toulá po daném systému, tak interaguje s prostředím až do místa, kde by teoreticky mělo dojít k její absorbci. Nicméně, nedojde k úplné absorbci, pouze se dojde k započítání do skórování a dále se jí srazí váha na nižší hodnotu. Nedojde tedy k zániku, a jedna částice může pokračovat v toulání se po systému. Pouze má nižší váhu, proto narazí-li na další absorbci, opět dojde k započtení do skórování, ale s nižší váhou, jelikož v tomto druhém místě by už částice neměla teoreticky existovat. Dojde-li po několika srážkách ke snížení váhy pod jistou hladinu, až tehdy dojde k celkovému zániku dané částice.\\

Tímto způsobem se dají zpřesnit výsledky za pomoci menšího počtu částic. Nicméně, jde o komplikovanější metodu a je potřeba znát širší matematický základ. V prvním případě je důležité určit, jakým zpsůobem se změní váha dané částice. Tato technika se nazývá \textbf{redukce variance} (redukce variance, jelikož zpřesňují výsledek a snižuji rozptyl, tedy varianci).

\subsection{Redukce variance}

První možností při sražení váhy je za pomoci teoretické absorbce. To se určí jednoduše dle vztahu:

$$ w = w' \left ( 1-\dfrac{\Sigma_\text{a}}{\Sigma_\text{t}} \right ). $$

Dále je třeba určit, při jaké energii dojde ke skutečnému zániku částice, jelikož trekování částice s příliš malou váhou je naopak kontraproduktivní. To se určuje dle tzv. Ruské rulety, jejíž nejjednodušší zápis je ve tvaru:

$$ P = 1-\dfrac{w}{w_0}, $$

kde $w_0$ značí váhu na začátku historie a $P$ značí pravděpodobnost, při které dojde k zániku.

Dále je možné podobnou metodu aplikovat na vznik částic (z hlediska neutronů štěpení (n,xn) reakce). V takovém případě se nová váha určí jako:

$$ w = w' \nu. $$

Tohle jsou jenom základní metody, ve skutečnosti je nutné aplikovat více metod redukce variance. Těmi hlavními jsou:

\begin{itemize}
  \item cut-off -- zánik při poklesu pod jistou hladinu (energetické, váhové, časové apod.),
  \item splitting -- rozdělení na 2 částice, každá s jinou váhou (geometrické, energetické, časové),
  \item ruská ruleta -- naopak zánik dané částice, kombinace se splittingem (velká váha splitting, malá váha ruská ruleta),
  \item biassing -- to je to zmiňované sražení váhy (absorbcí).
\end{itemize}

Nicméně nejdůležitější je pochopitelně kombinace všeho.

\include{kapitoly/chapter_08}
\section{Výpočty vyhoření jaderného paliva a příprava homogenizovaných dat pro celozónové výpočty}

\subsection{Transportní (deterministický) výpočet}

Ve zkratce: 

\begin{enumerate}
  \item[1.] Nejprve si vymyslím, jakou knihovnu dat použiji a pro tu šáhnu (např. \textbf{ENDF/B}. Jde o bodové knihovny a každá má různý počet bodů pro různé materiály).
  \item[2.] Tato nezpracovaná data je potřeba zpracovat (zgrupovat, upravit na teplotu apod.) vhodným nástrojem (např. \textbf{NJOY}), čímž vzniká pracovní knihovna obsahující stovky až tisíce grup.
  \item[3.] Následuje \textbf{Lattice výpočet} pomocí nějakého mikrokódu (např. \textbf{Newt/TRITON}, \textbf{Helios}), často v nekonečné mříži (pro tepelné systémy to stačí) pomocí transportní teorie. Tento model už by měl zohledňovat samostínění. Homogenizovat je možné i stochasticky bez aplikace transportní teorie (např. \textbf{Serpent}).
  \item[4.] Tím dochází k homogenizaci a k tvorbě homogenizovaných dat (difúzní koeficient, makroskopické průřezy apod.) pro daný palivový soubor. Zároveň se slučují grupy do menšího počtu (tepelný systém stačí 2, rychlé 4, lépe 8, záleží na složení. Neplechu dělá větší množství Pu, kde je potřeba více grup).
  \item[5.] Homogenizované veličiny putují do celozónového výpočtu makrokódem (např. \textbf{PARCS}, \textbf{Andrea}), který častou pouze s pomocí difúzního přiblížení určí hledané parametry (kritičnost, energetická distribuce, vyhoření apod.).
\end{enumerate}

\subsubsection{Homogenizace}

Pro deterministické kódy se postupuje podobně, jako při grupování mikroskopických průřezů, akorát že teď se grupují makroskopické průřezy vážené hustotou toku a nově i objemem. Opět se musí zachovat reakční rychlost:

\begin{equation}
  \boxed{
    \Sigma_G = \dfrac{\sum_{h \in G} \sum_i V_i \Sigma_{i,h} \Phi_{i,h}}{\sum_{h \in G} \sum_i V_i \Phi_{i,h}}.}
\end{equation}

Problém je, že takto homogenizovaná data jde použít pouze na původně řešený problém, nepaltí tedy obecně. Při změně geometrie se musí homogenizovat znovu. Pro následující nodální výpočet nám jde o:

\begin{itemize}
  \item $\Sigma_\text{tr}$ -- převrácená hodnota představuje dráhu, kterou neutron urazí po nekonečném množství srážek. Potřebujeme znát pro výpočet difúzního koeficientu, který se používá v difúzním přiblížení. Pro jeho určení se vychází z In-scatter nebo Out-scatter aproximace (nebo obojího), což je rozdíl totálního průřezu a sumy rozptylových průřezů z rychlé grupy.
  \item $\Sigma_\text{a}$ -- představuje zánik neutronů.
  \item $\nu \Sigma_\text{f}$ -- představuje produkci neutronů.
  \item $\kappa \Sigma_\text{f}$ -- představuje energii ze štěpení.
  \item $\Sigma_\text{s, g - g'}$ -- vyjadřuje přechod mezi grupami. 
  \item relativní výkony jednotlivých proutků.
\end{itemize}

Pro zachování spojitosti hustoty proudu mezi soubory se využívají tzv. ADF (Assembly Discontinuity Factors).\\

Homogenizace ve štěpném prostředí (speciálně pro tepelné reaktory, rychlé si myslím že jsou obtížnější) probíhá v nekonečné mříži za použití zrcadlových hraničních podmínek. To má v sobě některá úskalí:

\begin{itemize}
  \item Neuvažuje se únik, který tam ve skutečnosti je. Ten je možné opravit pomocí B1 aproximace, což je oprava přes hranice souboru na základě kritičnosti. Jednoduše řečeno, k totální grupové reakční rychlosti se přičte/odečte B-násobek grupové hustoty proudu tak, aby došlo k dosažení kritického stavu, čímž se získají zbylé homogenizované průřezy.
  \item Neuvažuje se prostorový gradient hustoty toku (v celozónovém výpočtu není hustota toku ve všech souborech středově souměrná). Oprava pomocí rehomogenizace, která kombinuje hustotu toku v nekonečné mříži a hustotu toku z celozónového výpočtu, která se získala po dosazení původních dat. Je důležité hlavně v blízkosti silných absorbátorů, kde je gradient hustoty toku významný.
  \item Spektrum (teď nemyslím prostorové, ale energetické) neodpovídá skutečnosti.
\end{itemize}

Homogenizace v neštěpném prostředí je obtížnější, protože tady už tuplem neznáme spektrum neutronů. Jsou zde jiné hraniční podmínky (vacuum v okolí reflektoru, reflective/periodic v okolí souborů), nepoužívá se B1 aproximace. Pro anizotropní rozptyl na vodíku se zavádí speciální opravná funkce, která se dá získat výpočtem v 1D prostředí pomocí In-scatter aproximace.


\subsubsection{Parametrizace}

Homogenizace se provádí pro různé stavy řešeného systému, tzv. odskoky. Celozónový výpočet totiž potřebuje data, která jsou závislá na provozu reaktoru (změní-li se teplota paliva, jak se změní homogenizovaná data). Odskoky se provádí např. na teplotě (moderátoru i paliva), vyhoření, hustotě (moderátoru) a koncentraci absorbátoru.\\

Odskoky fungují tak, že se změní zkoumaný parametr (stačí jednou nahoru a podruhé dolu), systém se zhomogenizuje a výsledná data se proloží lineárním (či jiným) fitem. Tak se získá závislost homogenizované veličiny na měnícím se parametru. Jde to i složitě, mohou se např. měnit 2 parametry najednou (což dle Pavla dává mnohem lepší výsledky), prokládání fitem se může lišit apod.\\

Mzi odskoky v klasické provozní fyzice patří:

\begin{itemize}
  \item teplota a hustota moderátoru $\dfrac{\partial \Sigma}{\partial T_\text{M}}$,
  \item teplota paliva $\dfrac{\partial \Sigma}{\partial \sqrt{T_\text{P}}}$,
  \item koncentrace kyseliny borité $\dfrac{\partial \Sigma}{\partial C_\text{B}}$,
  \item regulačních tyčí $\alpha$.
\end{itemize}

Takto získám původní provozní sadu průřezů, parametrizované odskoky a pro nové průřezy pak platí:

\begin{equation}
  \boxed{
    \Sigma (T_\text{M}, T_\text{P}, C_\text{B}, \text{CR}) = \Sigma + \dfrac{\partial \Sigma}{\partial T_\text{M}} \Delta T_\text{M} + \dfrac{\partial \Sigma}{\partial \sqrt{T_\text{P}}} \Delta \sqrt{T_\text{P}} + \dfrac{\partial \Sigma}{\partial C_\text{B}} \Delta C_\text{B} + \alpha \Delta \Sigma_\text{CR}.}
\end{equation}

\subsubsection{Celozónový výpočet}

Po takto zhomogenizovaných a zparametrizovaných souborech následuje celozónový výpočet. Ten probíhá buď pomocí zjednodušeného difúzního přístupu, nebo pomocí Simplified P3 metody transportní rovnice (SP3).\\

Obecně jsou 2 přístupy:

\begin{itemize}
  \item Diferenční přístup -- derivace se nahradí diferencemi a postupuje se pomocí sítí. Je to metodicky snažší, ale zdlouhavé. Objemové elementy jsou mezi sebou vázány přes hustotu toku (rovnost na hranici) a systém vede na soustavu rovnic, která je řešitelná triagonální maticí se 2 neznámými ($\Phi$ a $k_\text{ef}$). Nejprve se provede počáteční odhad a pak se iteruje, dokud se nedokonverguje k nějaké hodnotě $\Phi$ (vnitřní iterace), pomocí které se získá nový $k_\text{ef}$ (podělí se zdroj neutronů z iterace 0 a 1, vnější iterace). Vnější iterace končí po dosažení nějakého kritéria.
  \item Nodální přístup -- hustota toku se rozvine do bazických funkcí, a pak je možné redukovat objemové elementy až na úroveň celého souboru (nódy). Rychlejší, ale fakt složité.
\end{itemize}

Celozónové kódy musí být rychlé a jednoduché. Dále se nabízí coupling s termohydraulickým kódem (RELAP, TRACE), ale to je už fakt hardcore, protože nikdo pořádně neví, jak k tomu přistupovat.

\subsection{Výpočty vyhoření}
\include{kapitoly/chapter_10}
\section[Náhodná procházka]{Náhodná procházka při simulaci životního cyklu neutronů v jaderném reaktoru}

Jak bylo zmíněno, ýpočty ve stochatických kódech probhají díky simulaci tzv. \textbf{random walks - náhodné procházky} dané částice, tedy simulace skutečného života dané částice. K tomu je potřeba znát pár základů.

\subsection{Matematická vsuvka}

\subsubsection{Generace pseudonáhodných čísel}

Jde o stochastickou/pravděpodobnostní metodu, tedy základem je generování náhodných čísel. Jelikož ke generaci dochází strojově, nebude nikdy možné stanovit skutečně náhodná čísla, pouze tzv. pseudonáhodná. O to větší problém je vytvořit sekvenci náhodných/pseudonáhodných čísel tak, aby se neopakovaly. K tomu se ve většině kódů využívá tzv. \textbf{linear congruential random numbers generator}:

$$ S_{i+1} = ( S_i \cdot g + c ) \: \text{mod} \: 2^m, $$

kde $S_i$ značí seed (typicky nějaké velké číslo, pokaždé jiné), $g$ značí generátor/násobič (opět veliké číslo, to může být stejné), $c$ značí přírůstek (může být 0, 1, nebo opět jakékoliv) a z toho všeho určíme modulus (tedy zbytek po dělení) nějakého čísla $2^m$.

Přesnou kombinací těchto čísel je možné stanovit konkrétní generátor, např. MCNP využívá: $m = 63$, $g$ je různé a $c = 1$.

\subsubsection{Random Sampling - Náhodné vzorkování}

Dále je potřeba navzorkovat daná čísla podle jistých pravděpodobnostních funkcí $f(x)$. Generátor náhodných čísel nám poskytne uniformní rozdělení na intervalu (0,1), nicméně např. k rozptylu dochází pod jistým úhlem s danou pravděpodobností, nebo energie vzniklých neutronů také mají jistou pravděpodobnost.

Máme-li tedy danou pravděpodobnostní funkci (PDF) $f(x)$ normovanou na 1. Z té jsme schopni určit distribuční funkci (CDF) $F(x)$ jako postupný integrál:

$$ F(x) = \int_{-\infty}^x f(y) \text{d}y. $$

Označíme-li uniformní náhodná čísla jako $\xi$, ale my chceme získat/navzorkovat hodnoty $x$, pak tedy hledáme relaci mezi:

$$ F(x) = \xi \iff x = F^{-1}(\xi). $$

To je možné určit:

\begin{itemize}
  \item \textbf{direct sampling - přímé vzorkování}, tedy napřímo. Není možné vždy, může být zdlouhavé a koplikované (jednoduché např. u konstanty).
  \item \textbf{rejection sampling - metoda odmítnutí}, mnohem více univerzálmí metoda.
\end{itemize}

Nejprve se stanoví konstanta $c \geq 1$ a další hraniční PDE $g(x)$ kterou je možné navzorkovat napřímo (např. konstanta) tak, aby pro všechna $x$ platilo:

$$ f(x) \leq c g(x). $$

Nejprve navzorkuji danou $\bar{x}$ z funkce $g(x)$ (to je jednoduché, udělám to napřímo). Dále naleznu nějakou hodnotu $\xi$  jednotkového intervalu (0,1), a pokud platí, že:

$$ \xi \leq \dfrac{f(\bar{x})}{c g(\bar{x})}, $$

pak mohu s jistotou říci, že zároveň $\bar{x}$ je navzorkována i z původní $f(x)$. Je to v podstatě stejný způsob, jako když se integrauje přes Monte-Carlo. Vybírám hodnoty z intervalu (0,$cg(x)$), a pokud platí, že hodnota je zároveň v intervalu (0,$f(x)$), tak ji vezmu, jinak ne, viz obrázek \ref{rejection}.

\begin{figure}[H]
  \centering
  \includegraphics[width=0.7\textwidth]{img/rejection_sampling.png}
  \caption{Metoda odmítnutí.}
  \label{rejection}
\end{figure}

\subsection{Popis náhodné procházky}

Schematicky je náhodná procházka jednoduchá:

\begin{enumerate}
  \item[1.] Vznik -- částice nejprve vznikne, buď z externího zdroje, nebo za pomoci štěpení či (n,xn) reakce.
  \item[2.] Tracking -- částice se nějak pohybuje v daném prostředí. Sleduje se kudy prochází, jakými geometrickými hranicemi apod.
  \item[3.] Reakce -- tím, jak částice prochází prostředím, tak interaguje s okolní látkou, čímž se mění její směr, energie a v případě implicitní metody i váha
  \item[4.] Detektory/Tallies -- statistické vyhodnocení dané reakce.
  \item[5.] Zánik -- případné vyhodnocení za pomoci redukce variance.
\end{enumerate}

Toto se opakuje tisíckrát-milionkrát. 4ím větší geometrie, tím více částic je potřeba na detailnější prozkoumání celého systému. Na základě všech dat se poté vyhodnotí detektory, koeficient násobení, reakční rychlosti apod.

\subsubsection{Vznik}

Částice nejprve vznikne. Pokud máme externí zdroj, tak je dopředu nadefinováno energetické a směrové rozdělení. Pokud jde o zdroj ze štěpení, tak je třeba uvažovat energetické vzorkování přes štěpné spektrum a uniformní směrovou distribuci.

\subsubsection{Tracking}

Pokud znám počáteční energii a směr, je třeba určit, za jakou dobu dojde k další reakci (která se bude odehrávat na jiném místě). Opět se postupuje přes pravděpodobnostní PDE funkci ve tvaru:

$$ f(x) = \Sigma_\text{t} e^{-x \Sigma_\text{t}} $$

a distribuční CDF funkci:

$$ F(x) = 1 - e^{-x \Sigma_\text{t}}. $$

V tomto případě je CDF funkci možné vyjádřit pomocí přímého vzorkování jako:

$$ x = -\dfrac{1}{\Sigma_{\text{t}}} \text{ln}(1-\xi) = -\dfrac{1}{\Sigma_{\text{t}}} \text{ln}(\xi), $$

kde $\xi = F(x)$ navzorkujeme rovnoměrně z intervalu (0,1). 

Tohle platí, je-li geometrie složena homogenně pouze z jednoho materiálu. Máme-li ale geometrii s různými materiály (a tedy i různými $\Sigma_{t})$, je třeba postupovat postupně. Rozlišují se 2 přístupy:

\begin{itemize}
  \item Surface-tracking,
  \item Delta-tracking.
\end{itemize}

V případě \textbf{Surface-trackingu} se uplatňují izolované 2 výpočty před a za hranicí (tedy v materiálu 1 a 2). Má-li být zvdálenost mezi reakcemi $x$ delší, než je přímá trasa k hranici $d$ (tedy dojde-li k překročení hranice), musí se výpočet rozepsat, protože v novém materiálu už zbylé $x$ bude logicky jiné, kvůli jinému $\Sigma_\text{t}$. V novém materiálu tedy platí:

$$ e^{-x_2 \Sigma_2} = e^{-(x_1-d)\Sigma_2} $$

z čehož se vyjádří $x_2$ a pro celou dráhu $x$ bude platit:

$$ x = d + x_2 = d + (x_1 - d) \dfrac{\Sigma_1}{\Sigma_2}. $$

Opět, platí-li, že $x_2$ je opět větší než vzdálenost k další hranici, postupujeme pořád obdobně, dokud nevyjde, že $x_i$ skutečně skončí v daném materiálu. Hlavní filosofie tedy je, že se výpočet na každé hranici zastaví a podle aktuálního $\Sigma_\text{t}$ se přepočte vzdálenost mezi reakcemi $x$.

V případě \textbf{Delta-trackingu} se postupuje jinak, bez zastavování se na každé hranici. Cílem je stanovení virtuálního $\Sigma_0$, který by reprezentoval veškeré materiály nacházející se v daném směru, dle vztahu:

$$ \Sigma_0 = \Sigma_\text{m} - \Sigma_\text{t}, $$

kde $\Sigma_\text{m}$ značí maximální možný průřez, který se v systému vyskytuje (dává tedy horní odhad). Známe-li počáteční směr, určí se vzdálenost $x$ mezi reakcemi za pomocí pouze $\Sigma_\text{m}$ (tím získáme dolní odhad vzdálenosti), kde dojde k virtuální reakci. To, jestli dojde či nedojde k reakci, se určí dle pravděpodobnosti:

$$ P = \dfrac{\Sigma_0}{\Sigma_\text{m}} = 1 - \dfrac{\Sigma_\text{t}}{\Sigma_\text{m}}. $$

Pokud dojde k zamítnutí, proces pokračuje dál, dokud nedojde ke skutečné reakci. Výhody Delta-trackingu jsou patrné hlavně v momentu, kdy geometrie obsahuje spoustu ploch.

\subsubsection{Reakce}

Nyní jsme už v bodě, kde dojde k nějaké reakci, zbývá zjistit k jaké. To se opět určí pravděpodobnostně, tedy to, že dojde k $i$-reakci na $m$-izotopu určuje vztah:

$$ P = \dfrac{\Sigma_{i,m}}{\Sigma_\text{t}}. $$

V případě absorbce se dále může uplatnit redukce variance. V případě rozptylu se dále musí určit, pod jakým úhlem a s jakou energií se neutron rozptýlí. V případě štěpení kolik neutronů se emituje, atd. 

\subsubsection{Detektory}

V posledním případě zbývá zaznamenat dané reakce a určit nějakou integrální hodnotu. Může jít o hustotu toku, hustotu proudu, reakční rychlosti apod. Dá se postupovat analogově/implicitně, viz předešlá otázka.

Neméně důležité je zapracování nejistot. U výsledků se předpokládá Gaussovské rozdělení kolem nějaké střední hodnoty s daným rozptylem. Čím větší statistika, tím menší rozptyl. K tomuto určování se aplikují jakési matematicko-statistické nástroje, o kterých nemám ponětí, nicméně bavili jsme se o tzv. \textbf{Central Limit Theorem}, který tohleto umí dělat.

Výsledky jdou posléze i kombinovat mezi sebou. Dále se hodí vědět, že celkový rozptyl klesá s odmocninou celkové generace, tedy:

$$ \sigma^2 = \dfrac{\sigma^2_i}{N}. $$

\subsubsection{Zánik}

Nakonec neutron zanikne a už dále neexistuje :(
\section[Vyhořívání]{Výpočty vyhoření jaderného paliva, formulace úlohy a metoda prediktor-korektor}

UPRAVIT!!

\subsection{Výpočty vyhoření}

Vyhoření se počítá pomocí Batemanových rovnic:

\begin{equation}
  \dfrac{\text{d}n_i}{\text{d}t} = -\sigma_i \phi n_i + \sum_{j \to i} \gamma_i \sigma_j \phi n_j - \lambda_i n_i + \sum_{j \to i} \gamma_i \lambda_j n_j,
\end{equation}

nicméně aby šly dopočíst, je třeba znát hodnotu $\phi$, která se časem mění. Proto se postupuje pomocí tzv. metody \textbf{prediktor-korektor}, která nejprve odhadne $\phi$ na začátku cyklu, dopočte na konec cyklu (prediktor), přepočte novou $\phi$ z konce. Z těchto dvou hodnot poté stanoví průměr (korektor) a dopočte nové složení na konci cyklu.

Proto je důležité vhodně stanovit délku kroku. Například u tepelných systémů je třeba počáteční kroky volit velmi krátké, jelikož dochází k ustalování koncentrace xenonu, což má za následek velký nárůst $\phi$. U rychlých systémů to tak kritické není.

Nicméně, tato metoda je pouze na odhad hustoty toku. Stále máme kombinaci LDR o hodně členech (např. ENDF/B-VIII.0 1600 nuklidů). Vypíšu 2 metody řešení:

\subsubsection{MATREX metoda}

Maticovou formu Batemanových rovnic:

\begin{equation}
  \dfrac{\text{d}\vec{N}}{\text{d}t} = \textbf{A} \vec{N}(t)
\end{equation}

řeším ve formě exponenciály:

\begin{equation}
  \vec{N}(t) = \text{exp} (\textbf{A}t) \vec{N}(0),
\end{equation}

$$ \text{exp} (\textbf{A}t) = \sum_{k=0}^\infty \dfrac{(\textbf{A}t)^k}{k!}. $$

\subsubsection{CRAM metoda}

Neboli aproximace pomocí Chebyshevových racionálních funkcí. Obyčejně se využívá CRAM metoda 16. řádu, která má stejnou přesnost, jenom je rychlejší než MATREX metoda.

Funguje na stejném základu, pouze rozkládá exponenciální matici do Chebyshevových racionálních funkcí:

$$ \vec{N}(t) = \left [ \sum_{k=0}^\infty \dfrac{(\textbf{A}t)^k}{k!} \right ] \vec{N}(0) \approx \left [ \alpha_{0,K} \textbf{I} + 2 \text{Re} \left [ \sum_{j=1}^{K/2} \left ( -\dfrac{\alpha_{j,K}}{\theta_{j,K}} \right ) \left ( \textbf{I} + \textbf{A} \left ( -\dfrac{t}{\theta_{j,K}} \right ) \right ) ^{-1} \right ] \right ] \vec{N}(0), $$

kde $K$ představuje řád a $\alpha_{i,j}$ a $\theta_{i,j}$ jsou příslušné CRAMerovy koeficienty k dohledání v literatuře.
\include{kapitoly/chapter_13}
\section[Citlivostní analýza]{Zdroj nejistot numerických výpočtů v jaderných datech a analýza citlivosti koeficientu násobení}

% \section[Difúzní teorie]{Odvození a využití difuzní teorie v reaktorové fyzice}

\textbf{Difúzní rovnice} představuje základní rovnici stanovující prostorové rozložení neutronů v látce. Přesný popis neutronů je příliš komplikovaný, proto je vhodné volit určitá zjednodušení. Pro nejobecnější popis je nutné použít Boltzmannovu transportní rovnici, nicméně s určitými zanedbáními postačuje pouze difúuzní rovnice, případně difúzní rovnice s transportními korekčními faktory.

\subsection{Fickův zákon}

Difúzní rovnice je založena na \textbf{Fickově zákonu}, který je možné znát např. z chemie (šíření látky proti směru gradientu koncentrace):

\begin{equation}
    \boxed{ \textbf{J} = - D \text{grad} \Phi = -D \nabla \phi, }
\end{equation}

kde $\textbf{J}$ značí hustotu proudu neutronů (vektor), $D$ značí difúzni koeficient a $\Phi$ značí hustotu toku neutronů. To samé z části platí i pro neutrony. Pokud je hustota toku neutronů v jedné oblasti vyšší, neutrony putují opačným směrem. 

To a jak jsou neutrony ochotny se v dané látce šířít pak popisuje \textbf{difúzní koeficient} $D$. Difúzní rovnice předpokládá, že je difúzní koeficient funkcí materiálu, nikoliv polohy (izotropní rozptyl). V případě slabého anizotropního rozptylu je možné difúzní koeficient zavést za pomoci \textbf{střední volné dráhy pro transport} $\lambda_\text{tr}$ jako:

$$ D = \dfrac{\lambda_\text{tr}}{3}, $$

$$ \lambda_\text{tr} = \dfrac{1}{\Sigma_\text{tr}} = \dfrac{1}{\Sigma_\text{s}(1-\bar{\mu})}, $$

kde střední volná dráha pro transport udává průměrnou vzdálenost, kterou neutron urazí v původním směru po nekonečném množství rozptylových srážek. Pro izotropní rozptyl v těžišťové soustavě pak navíc platí:

$$ \bar{\mu} = \dfrac{2}{3A}. $$

$\bar{\mu}$ zde označuje cosinus úhlu rozptylu.

Obecně Fickův zákon platí všude, až na:

\begin{itemize}
    \item silně absorbční prostředí (absorbátory),
    \item vzdálenost menší než 3 střední volné dráhy pro transport od zdroje, nebo od hranic difúzního prostředí,
    \item silně anizotropní rozptyl.
\end{itemize}

Tyto podmínky jsou částečně všude, proto je potřeba k nim přistupovat opatrně, případně hledat jiná přiblížení.

\subsection{Odvození}

Odvození vychází z rovnice kontinuity. Pokud si představíme element objemu $V$, pak musí platit:

\begin{equation*}
    \boxed{
    \begin{pmatrix} \text{rychlost změny} \\ \text{počtu neutronů} \end{pmatrix} = \begin{pmatrix} \text{rychlost produkce} \\ \text{neutronů} \end{pmatrix} - \begin{pmatrix} \text{rychlost absorbce} \\ \text{neutronů} \end{pmatrix} - \begin{pmatrix} \text{rychlost úniku} \\ \text{(difuze) neutronů} \end{pmatrix}.
    }
\end{equation*}

Pokud $n(\textbf{r})$ označuje hustotu neutronů, pak \textbf{rychlost změny počtu neutronů} lze určit jako:

$$ \dfrac{\text{d}}{\text{d}t} \int_V n(\textbf{r}) \: \text{d} V, $$

přičemž derivaci je možné z nějaké chytré matematické věty šoupnout do integrálu, tedy:

$$ \boxed{\int_V \dfrac{\partial n}{\partial t} \: \text{d} V.} $$

\textbf{Rychlost produkce}, resp. \textbf{rychlost absorbce} je možné zapsat za pomoci reakčních rychlostí:

$$ \boxed{\int_V s(\textbf{r}) + \nu \Sigma_\text{f} \Phi(\textbf{r}) \: \text{d} V,} $$

$$ \boxed{\int_V \Sigma_\text{a} \Phi(\textbf{r}) \: \text{d} V.} $$

Rychlost úniku z elementu $V$ je pak dána difúzí a aplikací Fickova zákonu přes plochu $S$:

$$ \oint_S \textbf{J} \cdot \text{d} \textbf{S}, $$

což po aplikaci další chytré matematické věty pojmenované po nějakém slavném matematikovi přejde na:

$$ \boxed{\int_V \text{div} \textbf{J} \: \text{d} V.} $$

Teď už se to jenom poskládá dohromady, odstraní integrál (předpoklad, že to platí všude, tudíž se z integrální formy přejde do diferenciální), z divergence gradientu se udělá Laplacián (předpoklad, že $D$ není funkcí prostoru, a tudíž půjde vytkntout) a ještě se od hustoty neutronů přejde k hustotě toku neutronů, čímž dostaneme:

\begin{equation}
    \boxed{
        \dfrac{\partial n}{\partial t} = \dfrac{1}{v} \dfrac{\partial \Phi}{\partial t} = D \Delta \Phi - \Sigma_\text{a} \Phi + s + \nu \Sigma_\text{f} \Phi.
    }
\end{equation}

Ještě je možný jeden zápis, kdy difúuzní rovnici podělím difúzním koeficientem, čímž se do rovnice dostane difúzní délka $L$, resp. difúzní plocha $L^2$:

$$ L^2 = \dfrac{D}{\Sigma_\text{a}}, $$

\begin{equation}
    \boxed{
        \dfrac{1}{D v} \dfrac{\partial \Phi}{\partial t} = \Delta \Phi - \dfrac{1}{L^2} \Phi + \dfrac{s}{D} + \dfrac{\nu \Sigma_\text{f}}{D} \Phi.
    }
\end{equation}

\textbf{Difúzní plocha} popisuje látku vzhledem k tomu, kolik je neutron schopný procestovat v prostředí před tím, než bude absorbován. Pokud se bude uvažovat pravděpodobnost absorbce, která se zintegruje přes trasu a nalezne se střední hodnota tak vyjde, že $L^2$ představuje 1/6 střední hodnoty čtverce přímé zvdálenosti mezi místem vzniku a místem absorbce.

Hodnoty $L$ a $D$ se liší od materiálu, pro základní moderátory platí:

\begin{table}[h!]
    \centering
    \begin{tabular}{lccc}
    \toprule
    \textbf{Moderátor} & \textbf{hustota (g/cm$^3$)} & $D$ \textbf{(cm)} & $L$ \textbf{(cm)} \\ \midrule
    H\textsubscript{2}O & 1,00 & 0,13 & 2,62 \\
    D\textsubscript{2}O & 1,10 & 0,76 & 147 \\
    Be & 1,85 & 0,45 & 22 \\
    Grafit & 1,60 & 0,87 & 61 \\ \bottomrule
    \end{tabular}
    \caption{Tabulka vlastností moderátorů.}
\end{table}
    

\subsection{Řešení}

Jedná se o diferenciální rovnici, tudíž pro její řešení je nutné stanovit okrajové podmínky. Mezi ty patří:

\begin{itemize}
    \item Konečnost a nezápornost hustoty toku neutronů $0 \leq \Phi < \infty $.
    \item Rovnost hustoty toku a proudu na rozhraní $\Phi_A = \Phi_B$ \& $\textbf{J}_A = \textbf{J}_B$.
    \item Okrajová podmínka na vnějším rozhraní.
    \item Zdrojové podmínky.
\end{itemize}

Podmínku na vnějším rozhraní je možné zavést za pomoci \textbf{extrapolované délky} $d$, jelikož bylo zjištěno, že pokud hustota toku klesne na nulu ve vzdálenosti $d$ od rozhraní, pak se spočtené rozložení blíží realitě. Extrapolovanou délku je možné stanovit jako:

$$ d = 0,71 \lambda_\text{tr} = 2,13 D. $$

Zdrojové podmínky se týkají geometrie \textbf{zdroje} $s$. Jelikož difúzní rovnice neplatí v místě zdroje, je nezbytné tyto oblasti řešit samostatně. Postupuje se tak, že se difúzka řeší mimo zdroj, přičemž zdroj samotný se obklopí plochou, kterou prochází všechny emitované neutrony o vydatnosti $S$. Limitní podmínku je pak možné stanovit dle geometrie jako:

\begin{itemize}
    \item $ \lim_{x \to 0} 2 \: \textbf{J}(x) = S $ pro rovinný zdroj (2 plochy),
    \item $ \lim_{r \to 0} 4 \pi r^2 \: \textbf{J}(r) = S $ pro bodový zdroj (plocha sféry),
    \item $ \lim_{r \to 0} 2 \pi r \: \textbf{J}(r) = S $ pro přímkový zdroj (obvod kružnice).
\end{itemize}

Difúzka je jednoduše řešitelná i analyticky, kor stacionární případ. Stačí nalézt vhodného Laplaciána dle geometrie a přepsat rovnici do tvaru:

$$ \Delta \Phi \pm A^2 = 0 $$

Výsledné řešení je pak závislé na znamínku. Pokud převládá absorbce nad produkcí (ZAF1, difúze v prostředí $ - A^2$), vede řešení na hyperbolické funkce a Modifikované Besselovy funkce typu $I_n$ a $K_n$. Pokud převládá produkce nad absorbcí (ZAF2, homogenní reaktor $ + A^2$), vede řešení na goniometrické funkce a Besselovy funkce typu $J_n$ a $Y_n$.

\subsection{Grupová difúzní rovnice}

Difúzní rovnice je možné zapsat i s energetickou závislotí za pomoci rozgrupování. Poté se pro každou grupu řeší vlastní difúzka, přičemž přechod mezi grupami se zapisuje pomocí \textbf{rozptylového grupového průžezu} $\Sigma_{m \to n}$, tedy pravděpodobnost přechodu z grupy $m$ do grupy $n$. Grupa s nejvyšší energií se označuje indexem $1$ a s klesající energií roste číslo grupy. Skupinová difúzní rovnice pro grupu $g$ pak vypadá (stacionární případ, kde zdroj ze štěpení je zahrnut v koeficientu $s$):

\begin{equation}
    \boxed{
        D_g \Delta \Phi_g - \Sigma_\text{a,g} \Phi_g - \sum^N_{h=g+1} \Sigma_{g \to h} \Phi_g + \sum^{g-1}_{h=1} \Sigma_{h \to g} \Phi_h = -s_g.
    }
\end{equation}

Zde předpokládáme pouze downscattering, tedy přesun z vyšší grup do nižší grupy (resp. z nižšího pořadového čísla do vyššího). To obecně platí pro vysoké energie (neutrony vznikají s energií v řádech MeVů a postupně termalizují, tedy pouze ztrácejí energii rozptylovými srážkami), na úrovní tepelné energie se může objevovat i upscattering a celkový popis je složitější.

\subsection{2G přiblížení \& grupování}

Obecně je nutné uvažovat alespoň 2 grupy, tepelné a rychlé neutrony s hranicí na 0,625 eV. Ukazuje se, že hustotu toku tepelných neutronů je možné popsat za pomoci \textbf{Maxwellovy distribuce}. Pak je možné účinné průřezy vystředovat a uvažovat celou tepelnou oblast jako jednu grupu se středovanými průřezy. 

Maxwellova distribuce je dána:

$$ n(E) = \dfrac{2 \pi n}{(\pi k T)^{3/2}} \sqrt{E} \: \text{exp} \left(- \dfrac{E}{kT}\right), $$

rychlost je definuje:

$$ v(E) = \sqrt{\dfrac{2E}{m}}, $$

a pak je možné určit hustotu toku tepelných neutronů $\phi_T$ jako:

$$ \Phi_T = \int_T \Phi(E) \: \text{d}E = \int_T n(E) v(E) \: \text{d}E = \dfrac{2n}{\sqrt{\pi}} \sqrt{\dfrac{2kT}{m}} $$

Tím se určí střední hustota toku tepelných neutronů, která se pak aplikuje na středované průřezy:

$$ \bar{\Sigma} = \dfrac{1}{\Phi_T} \int_T \Sigma(E) \Phi(E) \: \text{d}E.$$

Pokud navíc platí oblast $1/v$ (a pokud ne, přidá se korekční faktor $g$), tak je možné nahradit:

$$ \Sigma(E) \Phi(E) = \Sigma(E_0) \Phi_0, $$

kde index 0 odpovídá energii 0,0253 eV, tedy 293 K, což se nalezne v tabulkách. Celkové výrazy pak budou:

$$ \boxed{\bar{\Sigma} = \dfrac{\sqrt{\pi}}{2} g(T) \Sigma(E_0) \sqrt{\dfrac{T_0}{T}},} $$

což platí pro všechny hodnoty (průřezy, energie, difúzní koeficienty apod.).

Nyní je už možné zkompletovat 2G sadu difúzních rovnic:

\begin{equation}
    \boxed{
        \Delta \Phi_T - \dfrac{1}{L_T^2} \Phi_T = -\dfrac{\Sigma_1 \Phi_1}{\bar{D}},
    }
\end{equation}

\begin{equation}
    \boxed{
        \Delta \Phi_1 - \dfrac{1}{\tau} \Phi_1 = -\dfrac{s_1}{D_1},
    }
\end{equation}

kde $\tau$ označuje Fermiho stáří neutronů (1/6 čtverce přímé vzdálenosti mezi vznikem a přechodem do tepelné grupy, příp. absorbce), a kde zdrojem tepelných neutronů je scattering z rychlé grupy, a zdroje v rychlé grupě jsou označeny jako $s_1$ (např. ze štěpení).

\subsection{Využití}

Difúzka se využívá jako první přiblížení, pokud nás zajímá transport neutronů, rozložení neutronového pole, kritičnost reaktoru, efekt moderátoru a reflektoru apod. Využívá se i v praxi, jelikož její řešení je velmi jednoduché. Např. pokud řešíme celozónový výpočet, tak většina všech full-core nástrojů využívají právě difuzní řešení (Andrea, PARCS apod.), které se řeší numericky (nodálně, sítěmi apod.). Z lattice kódu se vytáhne sada účinných průřezů a v full-core kódu se tyhle průřezy nastrkají do difúzní rovnice.

Mezi rovnice odvozené v rámci ZAF patří:

\subsubsection{Modifikovaná 2G kritická rovnice}

Vychází z předpokladu, že máme homogenní systém popsaný za pomoci 2G rovnice. Neutrony vznikají ze štěpení jako rychlé neutrony, přecházejí do tepelné oblasti kde inicializují štěpení. Celá úloha je pak zestaciarizována za pomoci koeficientu násobení. Ke štěpení rychlými neutrony nedochází vůbec.

$$ k_\text{ef} = \dfrac{k_\infty}{1 + B^2 \cdot M^2}, $$

kde $k_\infty$ představuje koeficient násobení pro systém bez úniku, $B^2$ představuje geometrický faktor (dáno geometrií) a $M^2$ představuje migrační plochu (dáno materiálem).

Na rovnici je možné pohlížet i jako na pravděpodobnost, že neutron neunikne $P_\text{NL}$. Pokud se navíc migrační plocha přepíše jako suma difúzní plochy (charakterizuje transport tepelných neutronů, tedy proces difuze) a Fermiho stáří (charakterizuje transport rychlých neutronů, tedy proces termalizace), pak platí:

$$ P_\text{NL} = \dfrac{k_\text{ef}}{k_\infty} = \dfrac{1}{1 + B^2 \cdot (L^2 + \tau)} \approx \dfrac{1}{1 + B^2 \cdot L^2} \cdot \dfrac{1}{1 + B^2 \cdot \tau} = P_\text{TNL} \cdot P_\text{FNL}. $$

Rovnici je možné i zobecnit pro konkrétnější případy, více grup apod.
% \section[Metody jader a vlastních funkcí]{Zavedení metody jader a metody vlastních funkcí pro řešení úloh z reaktorové fyziky a transportu záření}
% \section{Poruchová teorie a její využití v reaktorové fyzice}

\subsection{Odůvodnění}

Při provozu reaktoru dochází k častým dochylkám v parametrech od běžného stavu, tzv. \textbf{odchylky}. Ty se vyznačují:

\begin{itemize}
  \item prostorovou závislostí; jedná se o lokální změny v důsledku akumulace FP, lokálního vyhoření, vkládání vzorků k ozařování apod.
\end{itemize}

Pokud bychom chtěli systém analyzovat pomocí 2G, čí vícegrupového přístupu, nelze použít klasickou teorii, jelikož grupové konstanty se vztahují k celému objemu a nikoliv k lokálním změnám $\rightarrow$ bylo by potřeba velmi náročného 3D vícegrupového přístupu s prostorově závislými konstantami $\rightarrow$ vzhledem k výpočetní náročnosti nelze použít.\\

Alternativou je tzv. \textbf{poruchová teorie}, kterou lze využít pouze za předpokladu, že \textbf{daná porucha neovlivňuje výrazně hustotu toku neutronů v blízkosti poruchy}. Lze ji využít pro analýzu lokálních i celozónových poruch.

\subsection{Odvození poruchy reaktivity}

Uvažujme, že máme \textbf{holý homogenní tepelný reaktor} a hledáme poruchu v reaktivitě $\rho$. Pro koeficient násobení v takovém systému platí:

$$ k = \varepsilon p f \eta P_{NL} = \nu \left ( \dfrac{\Sigma_f}{\Sigma_a^F} \varepsilon p f P_{NL} \right ), $$

kde:

\begin{itemize}
  \item $\varepsilon$ (-) značí koeficient násobění rychlými neutrony,
  \item $p$ (-) značí pravděpodobnost úniku rezonančního záchytu,
  \item $f$ (-) značí koeficient využití tepelných neutronů $\left ( f = \dfrac{\Sigma_a^F}{\Sigma_a} \right )$,
  \item $\eta$ (-) značí regenerační faktor $\left ( \eta = \nu \dfrac{\Sigma_f}{\Sigma_a^F} \right )$,
  \item $P_{NL}$ (-) značí pravděpodobnost, že neutron neunikne ze systému,
  \item $\nu$ (-) značí průměrný počet neutronů uvolněných ze štěpení.
\end{itemize}

Označíme-li si:

$$  g = \left ( \dfrac{\Sigma_f}{\Sigma_a^F} \varepsilon p f P_{NL} \right ), $$

tak pro kritický systém musí platit, že:

$$ k = \nu g = 1, $$

kde parametr $\nu$ je pevně daný typem použitého paliva a vlivem poruchy se nemůže změnit, zatímco $g$ je závislý na složení a rozměrech systému a vlivem poruchy se změní. Nastane-li nějaká porucha, změní se koeficient násobení na $k'$, což vyústí na změnu v parametru $g'$, tedy:

$$ k' = \nu g' \neq 1. $$

Poté reaktivitu v systému s poruchou určíme podle vztahu:

$$ \rho = \dfrac{k'-1}{k} = \dfrac{k'-k}{k} = \dfrac{\cancel{\nu} g' - \cancel{\nu} g}{\cancel{\nu} g} = \dfrac{g'-g}{g}. $$

Naším cílem je vrátit reaktor zpět do kritického stavu. Předpokládejme, že nového kritického stavu jde docílit změnou parametru $\nu'$ (ačkoliv fyzikálně to možné není) tak, že:

$$ \nu' g' = 1. $$

Po vyjádření $g'$ a dosazení do předešlé rovnice získáme \textbf{vztah pro reaktivitu systému s poruchou jako}:

\begin{equation}
  \boxed{
  \rho = \dfrac{g'-g}{g} = \dfrac{\nu-\nu'}{\nu} = - \dfrac{\nu'-\nu}{\nu} = - \dfrac{\Delta \nu}{\nu}.
  \label{reaktivita_s_poruchou}}
\end{equation}

Chceme-li zjistit reaktivitu systému s poruchou, stačí pouze nalézt vyjádření pro $\Delta \nu$ (pouze matematická představa, fyzikálně nic takového neexistuje) a tu dosadit do vztahu \eqref{reaktivita_s_poruchou}.

\subsection{Způsoby výpočtu $\Delta \nu$}

\subsubsection{Matematický aparát}

\textbf{Operátorový zápis:}

Pro zjednodušení je vhodné vycházet z difúzní rovnice v operátorovém zápisu. Zavedeme si tzv. \textbf{difúzní operátor} $\mathcal{M}$ tak, že pro kritický systém musí platit:

\begin{equation}
  \boxed{
  \mathcal{M} \Phi = 0.
  \label{definice_maticovy_operator}}
\end{equation}

Jeho tvar je závislý na geometrii a počtu uvažovaných grup. Tvary mohou být následovné:

\begin{itemize}
  \item \textbf{1G v 1D:} \hspace{1.5cm} $\mathcal{M} = \dfrac{\text{d}}{\text{d}x} D(x) \dfrac{\text{d}}{\text{d}x} + F(x)$,
  \item \textbf{1G ve 3D:} \hspace{1.3cm} $\mathcal{M} = \text{div} D \text{grad} + F$,
  \item \textbf{2G ve 3D:} \hspace{1.3cm} $\mathcal{M} = \begin{pmatrix} \text{div} D_1 \text{grad} + F_{1,1} & F_{1,2} \\ F_{2,1} & \text{div} D_2 \text{grad} + F_{2,2} \end{pmatrix}$.
\end{itemize}

\textbf{Sdružený operátor:}

Dále je potřeba zavést \textbf{sdružený operátor} $\mathcal{M}^+$ tak, že pro každé 2 funkce $u$, $v$ splňující hraniční podmínky (nulovost na extrapolovaném rozhraní) platí:

\begin{equation}
  \boxed{
  \int_{reaktor} u \mathcal{M} v = \int_{reaktor} v \mathcal{M}^+ u.
  \label{definice_sdruzeny_operator}}
\end{equation}

Navíc, platí-li:

\begin{equation}
  \boxed{
  \mathcal{M} = \mathcal{M}^+,
  \label{definice_samosdruzeny_operator}}
\end{equation}

pak operátor $\mathcal{M}$ nazveme \textbf{samosdružený}.\\

Jednoduchým aplikováním metody per partes a s trochou důvtipu jde dokázat, že:

\begin{itemize}
  \item všechny 1G operátory jsou samosdružené,
  \item 2G operátor není samosdružený, ale platí $\mathcal{M}^+ = \mathcal{M}^T$ (tudíž je jednoduché ho vytvořit).
\end{itemize}

\textbf{Sdružená hustota toku:}

Jako \textbf{sdruženou hustotu toku}, neboli \textbf{funkci vlivu} definujeme funkci $\Psi$ tak, že funkce musí splňovat stejné okrajové podmínky jako funkce $\Phi$ (nulovost na extrapolovaném rozhraní) a zároveň rovnici:

\begin{equation}
  \boxed{
  \mathcal{M}^+ \Psi = 0.
  \label{definice_sdruzena_funkce}}
\end{equation}

Je tedy očividné, že pro 1G přiblížení platí $\Psi = \Phi$, ale pro 2G přiblížení už tato relace neplatí. Funkci $\Psi$ musíme v takovém případě dopočíst (např. numericky) pomocí vztahu \eqref{definice_sdruzena_funkce} (což ovšem není těžké, jelikož sdružený operátor $\mathcal{M}^+$ lze získat prostou transpozicí).\\
\\
Stejně jako u funkce $\Phi$, tak i u funkce $\Psi$ lze z hraničních podmínek určit pouze tvar, a nikoliv velikost (tu získáme z normování výkonu), nicméně pro poruchovou teorii nic takového není potřeba.

\subsubsection{1G poruchová teorie}

V 1G difúzní rovnici pro kritický reaktor platí vztah:

\begin{equation}
  \text{div} D \text{grad} \Phi + (\nu \Sigma_f - \Sigma_a) \Phi = 0,
  \label{1G_difuzka}
\end{equation}

což v operátorové notaci značí:

$$ \mathcal{M} \Phi = 0 \text{, kde:} $$
$$ \mathcal{M} = \text{div} D \text{grad} + (\nu \Sigma_f - \Sigma_a). $$

\textbf{a) Porucha v $\Sigma$:}

Nejprve uvažujme poruchu pouze v $\Sigma_f$ a $\Sigma_a$, přičemž difúzní koeficient $D$ zůstává zachován. Takovou poruchu lze vyjádřit pomocí malého přírůstku $\delta \Sigma$ jako:

$$ \Sigma_f' = \Sigma_f + \delta \Sigma_f, $$
$$ \Sigma_a' = \Sigma_a + \delta \Sigma_a. $$

Nyní uvažujme, že dojde k poruše (změní se $\Sigma'$ a $\Phi'$) a reaktor se opět uvede do kritického stavu změnou $\nu'$. V takovém (teď už novém stavu) musí opět platit rovnice \eqref{1G_difuzka} v označení:

$$ \mathcal{M'} \Phi' = 0 \text{, kde:} $$
$$ \mathcal{M'} = \text{div} D \text{grad} + (\nu' \Sigma_f' - \Sigma_a'). $$

Předpokládejme malou změnu v $\nu'$ jako:

$$ \nu' = \nu + \Delta \nu, $$

kterou spolu se změnami pro $\Sigma'$ dosadíme do operátoru $\mathcal{M'}$, upravíme a získáme:

$$ \mathcal{M'} = \mathcal{M} + \nu \delta \Sigma_f + \Delta \nu \Sigma_f + \Delta \nu \delta \Sigma_f - \delta \Sigma_a. $$

Dále zanedbáme člen $\Delta \nu \delta \Sigma_f$ (zajímá nás porucha pouze do 1. řádu) a výraz upravíme pomocí \textbf{poruchového operátoru} $\mathcal{P}$ na:

$$ (\mathcal{M} + \mathcal{P}) \Phi' = 0 \text{, kde:} $$
$$ \mathcal{P} = \nu \delta \Sigma_f + \Delta \nu \Sigma_f - \delta \Sigma_a. $$

Nyní zašátráme v paměti a vylovíme definici sdružené hustoty toku \eqref{definice_sdruzena_funkce}:

$$ \mathcal{M^+} \Psi = 0, $$

kterou spolu se zavedeným vztahem:

$$ (\mathcal{M} + \mathcal{P}) \Phi' = 0 $$

vynásobíme zleva $\Phi'$, resp. $\Psi$, zintegrujeme přes objem reaktoru $V$ a vzájemně odečteme (na pravé straně odečítám 0 od 0, což je stále 0), čímž dostaneme:

$$ \int_V (\Psi \mathcal{M} \Phi' - \Phi' \mathcal{M^+} \Psi) \text{d}V + \int_V \Psi \mathcal{P} \Phi' \text{d}V = 0. $$

Z definice sdruženého operátoru \eqref{definice_sdruzeny_operator} se nám první člen vynuluje, u druhého členu rozepíšeme poruchový operátor $\mathcal{P}$ a integrál rozsekáme přes sčítání, čímž získáme:

$$ \int_V \Psi \nu \delta \Sigma_f \Phi' \text{d}V + \int_V \Psi \Delta \nu \Sigma_f \Phi' \text{d}V - \int_V \Psi \delta \Sigma_a \Phi' \text{d}V = 0. $$

Nyní pouze vyjádříme hledané $\Delta \nu$, dosadíme do vztahu \eqref{reaktivita_s_poruchou} a získáme:

$$ \rho = - \dfrac{\Delta \nu}{\nu} = \dfrac{\int_V \Psi (\nu \delta \Sigma_f - \delta \Sigma_a) \Phi' \text{d}V}{\nu \int_V \Psi \Sigma_f \Phi' \text{d}V}. $$

Ještě využijeme předpokladu, že v lokálním místě poruchy se nám hustota toku příliš nemění (tedy $\Phi = \Phi'$) a faktu, že operátor $\mathcal{M}$ je samosdružený (tedy $\mathcal{M^+} = \mathcal{M}$). Čímž konečně získáváme finální vztah pro reaktivitu s poruchou:

\begin{equation}
  \boxed{
  \rho = \dfrac{\int_V (\nu \delta \Sigma_f - \delta \Sigma_a) \Phi^2 \text{d}V}{\nu \int_V \Sigma_f \Phi^2 \text{d}V}.
  \label{reaktivita_1G_Sigma}}
\end{equation}

Z výše uvedené rovnice tedy vyplývá, že \textbf{efekt poruchy makroskopických účinných průřezů lze získat jejím vážením přes druhou mocninu hustoty toku neutronů}.\\

Jako poruchu v účinných průřezech lze uvažovat například vkládání štěpného ($\Sigma_f$) či absorbčního ($\Sigma_a$) materiálu dovnitř AZ. Např. na VR-1, pokud zavádíme vzorek či detektor do AZ, musíme se v její úrovni pohybovat pomaleji, aby regulační tyče stihly kompenzovat přebytek reaktivity a ta nevzrostla nad bezpečnostní úroveň, při jejíž překročení se reaktor sám odstaví.\\

\textbf{b) Porucha v $D$:}

Nyní naopak předpokládejme poruchu v $D$ při zachování $\Sigma$, přičemž poruchu opět vyjádříme pomocí lineárního přírůstku:

$$D' = D + \delta D.$$

Následující postup je zcela identický, změní se nám pouze poruchový operátor na:

$$ \mathcal{P} = \text{div} \delta D \text{grad} + \Delta \nu \Sigma_f, $$

což po zintegrování dává:

$$ \int_V \Psi \mathcal{P} \Phi' \text{d}V = 0. $$

Nyní opět budeme uvažovat samosdruženost operátoru a minimální vliv poruchy na hustotu toku, dosadíme do \eqref{reaktivita_s_poruchou} a dostáváme:

$$ \rho = - \dfrac{\Delta \nu}{\nu} = - \dfrac{\int_V \Phi \text{div} \delta D \text{grad} \Phi \text{d}V}{\nu \int_V \Sigma_f \Phi^2 \text{d}V}. $$

Trošku začarujeme s operátorem divergence a gradientu a získáváme finální tvar:

\begin{equation}
  \boxed{
  \rho = - \dfrac{\int_V \delta D (\text{grad}\Phi)^2 \text{d}V}{\nu \int_V \Sigma_f \Phi^2 \text{d}V}.
  \label{reaktivita_1G_D}}
\end{equation}

Rovnice nám tedy říká, že \textbf{změny v difúzním koeficientu jsou váženy pomocí druhé mocniny změny hustoty toku}. Nárůst difúzního koeficientu vede k záporné změně reaktivity, protože se zvyšuje únik neutronů.\\

Efekt poruchy v difúzním koeficientu lze na VR-1 způsobit např. vypouštěním bublinek do AZ.\\

\textbf{c) Součet poruch:}

Pokud dochází k poruše v $D$ i $\Sigma$ současně, lze vzorce \eqref{reaktivita_1G_Sigma} a \eqref{reaktivita_1G_D} sečíst, čímž získáme ten nejfinálnější vztah:

\begin{equation}
  \boxed{
  \rho = \dfrac{\int_V [(\nu \delta \Sigma_f - \delta \Sigma_a) \Phi^2 - \delta D (\text{grad}\Phi)^2] \text{d}V}{\nu \int_V \Sigma_f \Phi^2 \text{d}V}.
  \label{reaktivita_1G}}
\end{equation}

Odvozené vztahy jsou použitelné pouze za předpokladu platnosti 1G difúzní rovnice. Toto přiblížení má ovšem své limity, jelikož nedokáže popsat hustotu toku neutronů v reflektoru a poruchy, které se odráží pouze v tepelných či rychlých neutronech. Dále je potřeba brát v potaz, že změny v $\Sigma_f$ a $\Sigma_a$ jsou vzájemně provázány.

\subsubsection{2G poruchový operátor}

Pro analýzu tepelných reaktorů je přesnější použít 2G přiblížení. Zde platí:

\begin{equation}
  \begin{matrix}
  \text{div} D_1 \text{grad} \Phi_1 - \Sigma_1 \Phi_1 + \nu \varepsilon \Sigma_{f2} \Phi_2 = 0, \\
  \text{div} D_2 \text{grad} \Phi_2 - \Sigma_2 \Phi_2 + p \Sigma_1 \Phi_1 = 0.
  \end{matrix}
  \label{1G_difuzka}
\end{equation}

Vyjdeme opět z operátorového zápisu, tedy:

$$ \mathcal{M} \Phi = 0 \text{, kde:} $$
$$ \mathcal{M} = \begin{pmatrix} \text{div} D_1 \text{grad} -\Sigma_1 & \nu \varepsilon \Sigma_{f2} \\ p \Sigma_1 & \text{div} D_2 \text{grad} - \Sigma_2 \end{pmatrix}. $$

\textbf{a) Porucha v $\Sigma$:}

Uvažujme nyní poruchu ve všech 3 $\Sigma$, vše ostatní zůstává zachováno. Platí:

$$ \Sigma_1' = \Sigma_1 + \delta \Sigma_1, $$
$$ \Sigma_2' = \Sigma_2 + \delta \Sigma_2, $$
$$ \Sigma_{f2}' = \Sigma_{f2} + \delta \Sigma_{f2}. $$

Do kritického stavu se reaktor vrátí opět změnou v $\nu'$ a $\Phi'$ a v systému s poruchou musí platit:

$$ \mathcal{M'} \Phi' = 0. $$

Opět vše podosazujeme do sebe (pozor, pracujeme po složkách, jde o matici), čímž vznikne nový poruchový operátor (logicky matice):

$$ \mathcal{P} = \begin{pmatrix} - \delta \Sigma_1 & \varepsilon(\Delta \nu \Sigma_{f2} + \nu \delta \Sigma_{f2}) \\ p \delta \Sigma_1 & - \delta \Sigma_2 \end{pmatrix}. $$

Dále aplikujeme úplně stejný přístup jako v 1G přiblížení (ale stále pracujeme po složkách) i předpoklad malé poruchy ($\Phi' = \Phi$). Jediné, co tentokrát nelze uvažovat je fakt, že $\Psi \neq \Phi$! Nakonec po tom všem dostáváme finální tvar:

\begin{equation}
  \boxed{
  \rho = \dfrac{\int_V \left ( - \delta \Sigma_1 \Psi_1 \Phi_1 + \varepsilon \nu \delta \Sigma_{f2} \Psi_1 \Phi_2 + p \delta \Sigma_1 \Psi_2 \Phi_1 - \delta \Sigma_2 \Psi_2 \Phi_2 \right ) \text{d}V}{\varepsilon \nu \int_V \Sigma_{f2} \Psi_1 \Phi_2 \text{d}V}.
  \label{reaktivita_2G_Sigma}}
\end{equation}

\textbf{b) Porucha v $D$:}

Uvažujeme-li naopak poruchu v $D$, tedy:

$$ D_1' = D_1 + \delta D_1, $$
$$ D_2' = D_2 + \delta D_2, $$

dostaneme finální vzorec tvaru:

\begin{equation}
  \boxed{
  \rho = \dfrac{\int_V \left ( - \delta D_1 \text{grad} \Psi_1 \cdot \text{grad} \Phi_1 - \delta D_2 \text{grad} \Psi_2 \cdot \text{grad} \Phi_2 \right ) \text{d}V}{\varepsilon \nu \int_V \Sigma_{f2} \Psi_1 \Phi_2 \text{d}V}.
  \label{reaktivita_2G_D}}
\end{equation}

\textbf{c) Součet poruch:}

Poruchy \eqref{reaktivita_2G_Sigma} a \eqref{reaktivita_2G_D} je možné opět kombinovat a výsledkem je absolutně nezapamatovatelný a šílený vztah, který se sem ani nevejde.\\

Jedinou neznámou nakonec představuje sdružená hustota toku $\Psi$, ale jak je psáno v úvodu, její získání není obtížné, stačí pouze řešit okrajovou úlohu \eqref{definice_sdruzena_funkce}, kde sdružený operátor $\mathcal{M^+}$ získáme transpozicí.\\

Výsledky rovnic \eqref{reaktivita_2G_Sigma} a \eqref{reaktivita_2G_D} nejsou překvapivé, jejich interpretace je stejná jako v 1G případě. Opět je třeba uvažovat propojení $\Sigma$ a především to, že $\Sigma_{f2}$ výrazně ovlivňuje $\Sigma_2$.

\subsection{Význam sdružené hustoty toku}

Na jeji hodnotu lze aplikovat stejná předpokládaná řešení jako na $\Phi$, zároveň musí splňovat stejné hraniční podmínky. Ve vícegrupovém přiblížení se většinou využívá numerických metod.\\

Uvažujme nyní 1G přiblížení. Mějme velmi malý absorbátor o objemu $V_p$, který po vložení do AZ vyvolá poruchu v bodě $r_0$, kterou lze vyjádřit pomocí Diracovy delta funkce jako:

$$ \delta \Sigma_a = \Sigma_{ap} \delta (r - r_0). $$

Pak po dosazení do rovnice \eqref{reaktivita_1G_Sigma} (ale ještě bez uvažování samosdruženosti operátoru) získáme vliv poruchy jako:

$$ \rho = - \dfrac{\int_V \Psi \delta \Sigma_a \Phi \text{d}V}{\nu \int_V \Psi \Sigma_f \Phi \text{d}V} = - \dfrac{\Sigma_{ap} V_p \Psi(r_0) \Phi(r_0) \text{d}V}{\nu \int_V \Psi \Sigma_f \Phi \text{d}V}. $$

Jmenovatel pro velikost poruchy je nezávislý na její povaze (tento integrál bude stejný pro libovolný bod $r$) a můžeme jej označit jako $1/C$. Pak, poku vyjádříme $\Psi(r_0)$, získáváme:

$$ \Psi_0 = - \dfrac{\rho}{C \Sigma_{ap} V_p \Phi(r_0)}. $$

Výraz $\Sigma_{ap} V_p \Phi(r_0)$ udává počet neutronů zachycených absorbátorem za 1 sekundu $\rightarrow$ potom hodnota $\Psi(r_0)$ \textbf{je úměrná záporné změně reaktivity vztažené na 1 absorbovaný neutron v bodě $r_0$ za 1 sekundu}.\\

Ukazuje se tedy, že hodnota funkce $\Psi(r_0)$ \textbf{měří význam bodu $r_0$ vzhledem k jeho vlivu na reaktivitu systému}. Proto se také sdružené hustotě toku říká funkce vlivu.\\

Podobný závěr je možné učinit také pro vložený štěpný materiál a pozitivní změnu reaktivity. \textbf{Sdružená hustota toku vyjadřuje relativní význam každé lokální změny v místě $r_0$, ať už vede k poklesu či vzrůstu reaktivity.}\\

Obdobné změny lze učinit i pro vícegrupové přiblížení. Ve 2G přiblížení je funkce $\Psi_1(r)$ úměrná změně reaktivity vyvolané poruchou způsobující změnu počtu neutronů v rychlé, a $\Psi_2(r)$ v tepelné grupě (přičemž index 1 označuje právě rychlou, a index 2 tepelnou grupu).\\

Z toho plyne finální dedukce, že \textbf{grupová sdružená hustota toku $\Psi_n(r)$ je úměrná vzrůstu či poklesu reaktivity v důsledku přírůstku či úbytku jednoho neutronu v dané grupě za sekundu v místě $r$}.

\subsection{Příklady aplikace poruchové teorie}

Poruchovou teorii je možné využít pro:

\begin{itemize}
  \item popis vlivu nerovnoměrné produkce FP (je třeba vyčíslovat numericky),
  \item analýzu částečně zasunuté regulační tyče (tzv. integrální charakteristiku regulační tyče):
\end{itemize}

\begin{equation}
  \rho(x) = \rho(H) \left [ \dfrac{x}{H} - \dfrac{1}{2 \pi} \sin \left ( \dfrac{2 \pi x}{H} \right ) \right ]
  \label{integralni_charakteristika}
\end{equation}

\begin{itemize}
  \item a derivací vztahu \eqref{integralni_charakteristika} relativní změnu vlivu regulační tyče (tzv. diferenciální charakteristiku regulační tyče):
\end{itemize}

\begin{equation}
  \dfrac{\text{d}\rho(x)}{\text{d}x} = \dfrac{\rho(H)}{H} \left [ 1 - \cos \left ( \dfrac{2 \pi x}{H} \right ) \right ].
  \label{diferencialni_charakteristika}
\end{equation}

% \section{Rovnice bodové kinetiky nulového reaktoru}

\subsection{Pojmy}

\textbf{Kinetika reaktoru} = zkoumá časové chování reaktoru se změnou vstupních parametrů.\\

\textbf{Vstupní parametry} = chápeme primárně $k_{\text{ef}}$, resp. $\rho$, a lze je ovlivnit změnou materiálů či geometrií systému.\\

\textbf{Výstupní parametry} = to, co v systému měříme ($P(t)$, $\Phi (\textbf{r}, t) $ atd.).\\

\textbf{Nulový reaktor} = neboli reaktor nulového výkonu; reaktor pracující v takovém výkonovém rozsahu, že jsou jeho zpětné vazby (ZV) zanedbatelné.

\begin{itemize}
  \item Výzkumné a energetické reaktory sem řadit nelze, jelikož se ZV projevují.
  \item Často složité odlišit, u některých nulových reaktorů lze pozorovat ZV (ve vyšších energetických hladinách) a naopak některé energetické reaktory lze provozovat bez ZV (při minimálním provozním výkonu).
\end{itemize}

\textbf{Zpětná vazba} = proces, díky kterému se změna výstupních parametrů ($P$, $\Phi$) může podílet na změnu vstupních parametrů.\\

\textbf{Dynamika reaktoru} = to samé co kinetika, pouze už uvažuje zapojení ZV.\\

\subsection{Odvození}

\textbf{Rovnice kinetiky reaktoru} = rovnice popisující závislost změny výstupních parametrů (výsledků) na změně vstupních parametrů.\\

K popisu lze využít transportní rovnici, resp. zjednodušenou difúzní rovnici $\rightarrow$ vede na komplikované soustavy, které nelze v obecném případě řešit analyticky (s projevem heterogenity systému).\\

Řešením jsou \textbf{Rovnice bodové kinetiky}, které zanedbávají změnu prostorového rozložení $\rightarrow$ nastane-li změna na vstupních parametrech (zvětší-li se reaktivita), tak změna výstupních parametrů (např. $\Phi$) se ve všech místech změní stějnou měrou $\rightarrow$ výstupní parametry se tedy pouze škálují a průběh zůstává zachován.\\

\textbf{Rovnice jednobodové kinetiky} = kromě prostorové závislosti se zanedbává i energetické rozdělení $\rightarrow$ vede na 1G rovnice.\\

V reálu to tak není, ale kupodivu dávají rovnice přijatelné výsledky.\\

Pro odvození se vychází z 1G difúzní rovnice (s konstantním $D$ a $\Sigma_a$):

\begin{equation}
  \boxed{
  \dfrac{\partial n(\textbf{r}, t)}{\partial t} = D \Delta \Phi (\textbf{r}, t) - \Sigma_a \Phi (\textbf{r}, t) + Q (\textbf{r}, t).
  \label{difuzka}}
\end{equation}

\subsubsection{a) Bez vlivu zpožděných neutronů:}

Uvažuji zjednodušení tvaru:

$$ Q (\textbf{r}, t) = k_\infty \Sigma_a \Phi (\textbf{r}, t), $$

$$ L^2 = \dfrac{D}{\Sigma_a}, $$

$$ B_m^2 = \dfrac{k_\infty - 1}{L^2}, $$

$$ n(\textbf{r}, t) = \dfrac{\Phi (\textbf{r}, t)}{v}$$

a předpokládáme, že rovnici \eqref{difuzka} lze řešit metodou separace proměnných, tedy:

$$ \Phi (\textbf{r}, t) = \Psi (\textbf{r}) \cdot T(t). $$

Poté rovnice \eqref{difuzka} vede na rovnici:

\begin{equation}
  v D \left ( \dfrac{\Delta \Psi (\textbf{r})}{\Psi (\textbf{r})} + B_m^2 \right ) = \dfrac{1}{T(t)} \dfrac{d T(t)}{d t} = \text{konst.} = - \omega,
  \label{difuzka_v_separaci}
\end{equation}

tedy na 2 obyčejné diferenciální rovnice provázané konstantou $\omega$.\\

Rovnice s $\Psi (\textbf{r})$ vede po uvažování okrajových podmínek (extrapolované rozhraní, konečnost, spojitost apod.) na vlastní funkce, jejichž tvar závisí na použité geometrii a tvaru Laplaciánu (kombinace goniometrických, Besselových, hyperbolických apod.). Řešení vyplývá z jednoduché vlnové rovnice:

$$ \Delta \Psi (\textbf{r}) + B_n^2 \Psi (\textbf{r}) = 0, $$

kde vztah mezi \textbf{vlastními čísly} $B_n$ a \textbf{materiálovým faktorem} $B_m$ je svázán pomocí určené konstanty $\omega$ jako:

$$ \omega_n = vD \cdot (B_n^2-B_m^2). $$

Obecně lze výsledek zapsat tvarem:

$$ \Psi (\textbf{r}) = \sum_n A_n \Psi_n (\textbf{r}), $$

kde $A_n$ značí normalizační konstantu a zjistíme ji z výkonu reaktoru.\\

Rovnice s $T(t)$ vede na exponenciálu tvaru:

$$ T(t) = C e^{- \omega t}. $$

Jelikož je ovšem $\omega$ závislá na volbě vlastních čísel, tak i zde platí superpozice a celkovou hustotu toku neutronů $\Phi (\textbf{r}, t)$ spočteme přes sumu všech vlastních funkcí jako:

\begin{equation}
  \Phi (\textbf{r}, t) = \sum_n A_n \Psi_n (\textbf{r}) e^{- \omega_n t}.
  \label{difuzka_reseni}
\end{equation}

Jelikož vlastní čísla splňují bilanci $B_1 < B_2 < B_3 < ...$, platí to samé i pro $\omega_1 < \omega_2 < \omega_3 < ...$ a první vlastní číslo po chvíli převáží ta zbylá. Proto dále zavádíme \textbf{geometrický faktor} $B_g$ jako první nejmenší vlastní číslo, tedy $B_g = B_1$.\\

Pro stacionární systém navíc platí $\omega = 0$ a poté $B_m = B_g$ (viz ZAF2).\\

Nyní přejdeme k prostorové nezávislosti (což je vlastně smysl celé kapitoly :D). Lze uvažovat (za předpokladu převážení prvního členu v rovnici \eqref{difuzka_reseni}), že:

$$ \Phi (\textbf{r}, t) \doteq v n(t) \Psi_1 (\textbf{r}) $$

a hustota neutronů $n(t)$ je zároveň úměrná maximální hustotě toku v soustavě (předpoklad rovnice jednobodové kinetiky), tedy:

$$ n(t) \doteq \text{konst.} \cdot \Phi_{max} (t). $$

Po dosazení do rovnice \eqref{difuzka_v_separaci} získáme novou rovnici tvaru:

\begin{equation}
  v D \left ( \dfrac{\Delta \Psi_1 (\textbf{r})}{\Psi_1 (\textbf{r})} + B_m^2 \right ) = \dfrac{1}{n(t)} \dfrac{d n(t)}{d t} = \text{konst.} = - \omega_1,
  \label{rovnice_kinetiky_v_separaci}
\end{equation}

která opět vede na 2 obyčejné diferenciální rovnice provázané konstantou $\omega_1$. Nyní už ovšem nejde o superpozici, jelikož uvažujeme pouze první člen (ačkoliv nestacionaritu zachováváme).\\

Pro zopakování a osvěžení paměti, stále platí:

$$ B_g = B_1, $$

$$ \omega_1 = vD \cdot (B_g^2 - B_m^2). $$

Zavedeme novou veličinu $\ell$ (s) jako \textbf{střední dobu života neutronů} vztahem:

\begin{equation}
  \boxed{
  \ell \equiv \dfrac{1}{v \Sigma_a} \dfrac{1}{1+L^2B_g^2}
  \label{stredni_doba_zivota}}
\end{equation}

a připomeneme si 1G rovnici pro stacionární reaktor:

$$ k_{\text{ef}} = \dfrac{k_{\infty}}{1 + L^2 B_g^2}. $$

Z těchto dvou vztahů lze vyjádřit parametr $\omega_1$ (důkaz dosazením) jako:

$$ \omega_1 = -\dfrac{k_{\text{ef}} - 1}{\ell}, $$

Což lze dosadit do rovnice \eqref{rovnice_kinetiky_v_separaci} (část s $\Psi$ už nemusím řešit) a získáváme \textbf{Rovnici jednobodové kinetiky}:

\begin{equation}
  \boxed{
  \dfrac{dn(t)}{dt} = \dfrac{k_{\text{ef}} - 1}{\ell} n(t).
  \label{rovnice_kinetiky_reseni}}
\end{equation}

Tím jsme si odvodili obyčejnou diferenciální rovnici 1. řádu pro hustotu neutronů $n(t)$, kterou lze řešit jednoduše pomocí integračního faktoru/separace proměnných (čímkoliv).\\

Často nás ale více než hustota neutronů zajímá časový vývoj výkonu, tedy $P(t)$. Zde ale platí úměra, tudíž hustotu neutrnonů je možné nahradit výkonem.\\

K rovnici \eqref{rovnice_kinetiky_reseni} je možné dojít i jednoduchou úvahou. Jelikož platí úměra mezi $n(t)$ a $N(t)$:

$$ n(t) \sim N(t), $$

lze vycházet právě z počtu neutronů v jedné generaci. Pro přírůstek mezi generacemi totiž platí:

$$ dN = k_{\text{ef}}N - N, $$

což po vydělení časem $dt$ na LS rovnice, resp. dobou života jedné generace $\ell$ na PS rovnice spěje k tíženému řešení:

$$ \dfrac{dN(t)}{dt} = \dfrac{k_{\text{ef}} - 1}{\ell} N(t). $$

Dále je možné rovnici \eqref{rovnice_kinetiky_reseni} přepsat pomocí reaktivity $\rho$. K tomu si zavedeme \textbf{střední dobu vzniku neutronů} $\Lambda$ (s) jako:

\begin{equation}
  \boxed{
  \Lambda \equiv \dfrac{\ell}{k_{\text{ef}}}.
  \label{stredni_doba_vzniku}}
\end{equation}

Po lehké úpravě, usměrnění rovnice \eqref{rovnice_kinetiky_reseni} a úvaze, že $\Lambda = \text{konst.}$, dostáváme nový výraz pro rovnici jednobodové kinetiky:

\begin{equation}
  \boxed{
  \dfrac{dn(t)}{dt} = \dfrac{\rho (t)}{\Lambda} n(t).
  \label{rovnice_kinetiky_reaktivita}}
\end{equation}

$\Lambda$ v podstatě vyjadřuje dobu, za kterou se zreprodukuje 1 neutron. Platí tedy:

\begin{itemize}
  \item $k_{\text{ef}} > 1$ $\rightarrow$ $\Lambda < \ell$ $\rightarrow$ nadkritický systém a tedy neutrony se zreprodukují rychleji, než je doba jejich života,
  \item $k_{\text{ef}} < 1$ $\rightarrow$ $\Lambda > \ell$ $\rightarrow$ podkritický systém a zreprodukování neutronu trvá déle, než doba jejich života.
\end{itemize}

\textbf{Perioda reaktoru:}

Rovnice udávají, jak rychle se mění výkon v systému v závislosti na $k_{\text{ef}}$ a $\ell$. Zadefinujeme si \textbf{periodu reaktoru} $T_e$ (s) jako dobu, za kterou se výkon v systému změní e-krát, pomocí vztahu:

\begin{equation}
  \boxed{
  T_e \equiv \dfrac{\ell}{k_{\text{ef}} - 1}.
  \label{perioda}}
\end{equation}

Zatímco $k_{\text{ef}}$ lze ovlivnit (geometrie, obohacení, materiály), $\ell$ je pevně dáno a spjato se systémem\footnote{Teoreticky to lze také ovlivnit, ale asi těžko z rychlého reaktoru udělám tak jednoduše lehkovodní, žejo.}.\\

Obecně platí, že rychlé reaktory mají střední dobu kratší (nedochází k moderaci, řádově 10$^{-7}$) než moderované lehkou vodou (10$^{-5}$) či grafitem (10$^{-3}$). Rychlý reaktor tedy bude na změny $k_{\text{ef}}$ reagovat mnohem rychleji, než reaktor moderovaný grafitem.\\

\subsubsection{b) S vlivem zpožděných neutronů:}

Nejprve si ujasníme, o co se jedná. Neutrony vznikající při štěpení můžeme členit na:

\begin{itemize}
  \item \textbf{Okamžité neutrony} -- vznikají ihned (do $10^{-13}$~s) emisí z mateřského jádra (FP -- Fission Product) se střední energií cca 2~MeV. Při štepení se FP nacházejí v excitovaném stavu a s přebytkem neutronů $\rightarrow$ těch se mohou zbavit buď za pomoci $\beta^-$ rozpadu (vzniká dceřiné jádro), nebo emisí okamžitého neutronu. Často se tyto FP označují jako \textbf{prekurzory}.
  \item \textbf{Zpožděné neutrony} -- jedná se o neutrony, které se uvolňují až po nějaké době, se střední energií cca 0,5~MeV. Vznikají emisí neutronů z dceřiných jader (DP -- Daughter Product), které vznikají radioaktivním rozpadem FP. DP se často označují jako \textbf{emitory}.
\end{itemize}

Ačkoliv jsou zpožděné neutrony emitovány emitory, pro jejich charakteristiku je přiřazujeme původním prekurzorům. Těch mohou být desítky, proto je dělíme do několika skupin (JEFF 8~skupin, ENDF/B 6~skupin) podle poločasu rozpadu. Pro popis se zavádí tzv. \textbf{podíl zpožděných neutronů} $\beta$ (-) jako:

\begin{equation}
  \boxed{
  \beta \equiv \dfrac{\nu_D}{\nu_T},
  \label{zpozdenky}}
\end{equation}

kde:

\begin{itemize}
  \item $\nu_D$ (-) značí střední počet zpožděných neutronů vzniklých při jednom štěpení,
  \item $\nu_T$ (-) značí střední počet všech vzniklých neutronů.
\end{itemize}

Obdobně lze zavést $\nu_i$, $\beta_i$, $T_{1/2}^i$, $\tau_i$ a $\lambda_i$ pro jednotlivé skupiny (rodiny) zpožděných neutronů.\\

Dále se pro popis zavádí tzv. \textbf{efektivní střední doba života} $\ell^*$ (s) jako:

\begin{equation}
  \boxed{
  \ell^* \equiv \ell(1-\beta) + \sum_i \beta_i \tau_i,
  \label{efektivni_stredni_doba_zivota}}
\end{equation}

\textbf{efektivní střední doba vzniku} $\Lambda^*$ (s) jako:

\begin{equation}
  \boxed{
  \Lambda^* = \dfrac{\ell^*}{k_{\text{ef}}}
  \label{efektivni_stredni_doba_vzniku}}
\end{equation}

a \textbf{efektivní perioda reaktoru} $T_e^*$ (s) jako:

\begin{equation}
  \boxed{
  T_e^* \equiv \dfrac{\ell^*}{k_{\text{ef}} - 1}.
  \label{efektivni_perioda}}
\end{equation}

V podstatě se jedná o vážený průměr přes koeficienty $\beta$, a ačkoliv je podíl zpožděných neutronů minimální (do 1~\%), díky dlouhým $\tau$ se efektivní doba života velmi prodlouží a perioda reaktoru natáhne. Proto jsou zpožděné neutrony velmi důležité k řízení reaktoru. Nastane-li kritičnost na okamžitých neutronech, tento prodlužovací efekt zcela vymizí, perioda reaktoru se zkrátí až o několik řádů a máme tu druhý Černobyl.\\

Pro klasický PWR reaktor platí, že $\sum \beta_i \tau_i \approx 0,1$~s. S uvažováním zpožděných neutronů se efektivní perioda natáhne o několik řádů a reaktor už není tolik citlivý na změnu $k_{\text{ef}}$.\\

Lépe se řídí takové systémy, které mají větší $\beta$. Ve skutečnosti není reaktor takto ideální. Je třeba dále započítávat fotoneutrony (vznikající ($\gamma$,n) reakcí na lehkých jádrech, např. Be), více skupin zpožděných neutronů apod. Důležitá je i energetická závislost. Jelikož zpožděné neutrony vznikají s menší energií (0,5~MeV vs. 2~MeV) a mají náskok ve zpomalování. Díky nižší energii nemohou zpožděné neutrony nikdy zapříčinit štepení na štěpitelných jádrech.\\

Kvůli tomu všemu se zavádí tzv. \textbf{efektivní podíl zpožděných neutronů} $\beta_{\text{ef}}$ (-), což je umělá hodnota, která koriguje energetický rozdíl ve skupinách, jelikož každá ze skupin má jiný vliv na štepení. Lze ji zavést pomocí vztahu:

\begin{equation}
  \beta_{\text{ef}} = \beta \cdot I,
\end{equation}

kde $I$ (-) značí tzv. \textbf{funkci vlivu} a závisí na konkrétním reaktoru. Říká, jak je snadné pro zpožděné neutrony štěpit, oproti okamžitým neutronům. Obecně se pohybuje okolo $\approx 1$, při bližším studiu lze napsat: FR $<1$ a LWR $>1$.\\

Nyní si ještě ukažme rovnice jednobodové kinetiky se zpožděnými neutrony. Odvození je podobné jako v předcházejícím případě, pouze se původní vztah modifikuje. Výsledkem je soustava lineárních diferenciálních rovnic v destrukčním tvaru:

\begin{equation}
  \boxed{
  \dfrac{dN}{dt} = \dfrac{k_{\text{ef}}(1-\beta_{\text{ef}})-1}{\ell} N(t) + \sum_{i=1}^m \lambda_i C_i(t),
  \label{rovnice_kinetiky_zpozdenky_1}}
\end{equation}

\begin{equation}
  \boxed{
  \dfrac{dC_i}{dt} = -\lambda_i C_i(t) + \dfrac{\beta_{\text{ef},i} k_{\text{ef}} N(t)}{\ell},
  \label{rovnice_kinetiky_zpozdenky_2}}
\end{equation}

resp. rovnice v produkčním tvaru:

\begin{equation}
  \boxed{
  \dfrac{dN}{dt} = \dfrac{\rho - \beta_{\text{ef}}}{\Lambda} N(t) + \sum_{i=1}^m \lambda_i C_i(t),
  \label{rovnice_kinetiky_zpozdenky_3}}
\end{equation}

\begin{equation}
  \boxed{
  \dfrac{dC_i}{dt} = -\lambda_i C_i(t) + \dfrac{\beta_{\text{ef},i}  N(t)}{\Lambda}.
  \label{rovnice_kinetiky_zpozdenky_4}}
\end{equation}

Druhý člen v první rovnici vždy vyjadřuje přírůstek neutronů rozpadem prekurzorů $C$, přičemž každý prekurzor je vázán vlastním rozpadovým zákonem (druhá rovnice), ke kterému se přičítá vznik ze štěpení.

\subsection{Přehled vzorečků}

Na závěr rychlá vzorečkiáda:

\begin{table}[H]
\centering
\caption{Vzorečky s rovnicemi jednobodové kinetiky.}
\label{table_vzorecky_kinetika}
\begin{tabular}{@{}rcc@{}}
\toprule
\textbf{Parametr}                 & \textbf{Bez zpožděnek} & \textbf{Se zpožděnkami}   \\ \midrule
\textbf{Střední doba života}      & $\ell = \dfrac{1}{v \Sigma_a} \dfrac{1}{1+L^2B_g^2}$    & $\ell^* = \ell(1-\beta) + \sum_i \beta_i \tau_i$      \\ [15pt]
\textbf{Střední doba vzniku}      & $\Lambda = \dfrac{\ell}{k_{\text{ef}}}$                 & $\Lambda^* = \dfrac{\ell^*}{k_{\text{ef}}}$           \\ [15pt]
\textbf{Perioda reaktoru}         & $T_e = \dfrac{\ell}{k_{\text{ef}} - 1}$                 & $T_e^* = \dfrac{\ell^*}{k_{\text{ef}} - 1}$           \\ [15pt]
\textbf{R-ce v destrukčním tvaru} & $\dfrac{dN}{dt} = \dfrac{k_{\text{ef}} - 1}{\ell} N(t)$ & $\dfrac{dN}{dt} = \dfrac{k_{\text{ef}}(1-\beta_{\text{ef}})-1}{\ell} N(t) + \sum_{i=1}^m \lambda_i C_i(t)$        \\ [15pt]
                                  &                                                         & $\dfrac{dC_i}{dt} = -\lambda_i C_i(t)+\dfrac{\beta_{\text{ef},i} k_{\text{ef}} N(t)}{\ell}$           \\ [15pt]
\textbf{R-ce v produkčním tvaru}  & $\dfrac{dN}{dt} = \dfrac{\rho}{\Lambda} N(t)$           & $\dfrac{dN}{dt} = \dfrac{\rho - \beta_{\text{ef}}}{\Lambda} N(t) + \sum_{i=1}^m \lambda_i C_i(t)$                 \\ [15pt]
                                  &                                                         & $\dfrac{dC_i}{dt} = -\lambda_i C_i(t) + \dfrac{\beta_{\text{ef},i}  N(t)}{\Lambda}$                            \\ [15pt] \bottomrule
\end{tabular}
\end{table}


% \include{chapter_05}
% \include{chapter_06}
% \section{Prinicipy metody Monte-Carlo}

\subsection{Základní filosofie}

Základním principem pro deterministické kódy je přímé řešení matematických rovnic popisujících chování v jaderném reaktoru, tedy difúzní rovnice, nebo transportní rovnice. Máme tedy jasný matematický model a ten řešeíme pomocí numeriky.\\

Oproti tomu stochastické metody ke svému řešení žádné rovnice nepotřebují a simulují přímo konkrétní částice. Modelování tedy probíhá na takovém prinicpu, že se jasně stanoví geometrie a pomocí generátoru náhodných čísel se generují imaginární částice, které s danými geometriemi interagují. To, jak budou částice interagovat čistě závisí na účinných průřezech (je potřeba znát diferenciální průřezy, jelikož nás i zajímá směr příletu/odletu částice). \\

Z této základní filosofie tedy platí jasné rozdíly mezi deterministickými a stochastickými metodami:\\

Deterministické:
\begin{itemize}
  \item Rychlejší, ale přesné pouze tak, jak přesný je daný matematický model.
  \item Získám jeden výsledek bez nejistot (ačkoliv nejistoty jsou dané zjednodušeními, není je tak jednoduché vyčíslit).
  \item Složité modelování geometrie $\rightarrow$ aplikují se různá geometrická zjednodušení (nekonečná mříž, 2D aproximace apod.)
  \item Účinné průřezy se musejí grupovat $\rightarrow$ je potřeba více programů a více přístupů (nejprve příprava v rámci jednoho souboru pomocí transportky, poté aplikace ve velkém celozónovém modelu pomocí difúzní rovnice).
\end{itemize}

Stochastické:
\begin{itemize}
  \item Pomalejší, ale teoreticky přesnější. S rozvojem techniky se dá očekávat převládnutí. Přesnost je ovlivněna hlavně účinnými průřezy, a pak trochu generací pseudonáhodných čísel.
  \item Vždy mi nástroj vyplivne výsledek s danou nejistotou, kterou je teoreticky možné snížit simulací většího počtu částic.
  \item Jednoduché 3D modelování.
  \item Spojité účinné průřezy (lze získat pomocí NJOY).
\end{itemize}

Stochatické metody tedy nepotřebují žádné matematické rovnice, celý výpočet probíhá pomocí tzv. \textbf{náhodné procházky} metodou Monte-Carlo, což je dále rozebráno v další otázce.\\

Jako částice je možné simulovat absolutně cokoliv, záleží na znalosti účinných průřezů a rozšíření výpočetního kódu:
\begin{itemize}
  \item Serpent 2 -- hlavní využití pro reaktorovou fyziku, specializuje se tedy převážně na neutrony (ačkoliv by měl nějak zohledňovat i (n,$\gamma$) reakce, ze zkušenosti se mi zdá, že to moc nefunguje).
  \item MCNP 6 -- momentálně asi nejznámnější kód, umí modelovat neutrony, fotony, elektrony..., skoro cokoliv. Bere se jako benchmark pro ostatní kódy, ale je složitý na syntaxi a zdlouhavý.
  \item OpenMC -- o tom moc nevím. Dobrý je, že je zadarmo.
  \item KENO -- low-budget alternativa od Scaleu. Prý stojí za prd, ale umí hezky spolupracovat s ostatními SCALE balíky, tudíž se taky může hodit.
\end{itemize}

\subsection{Analogová a neanalogová metoda}

Základním principem \textbf{analogové metody} je to, že trekujeme konkrétní neutrony, které když zaniknou (absorbcí či únikem), tak prostě zaniknou a dál neexistují. Při výpočtu skórování v datém detektoru/tallies se tedy započítávají s váhou 1. Tato metoda více odpovídá realitě, nicméně je zdlouhavá, jelikož pro přesnější výsledek je nutné simulovat větší populaci částic.\\

To platí např. pro prostředí s velkou absorbcí a malým zdrojem (chceme-li např. řešit problematiku stínění, reflektoru apod.)\\

Oproti tomu \textbf{neanalogová metoda} či \textbf{implicitní metoda} funguje tak, že narodí-li se částice, tak je jí přiřazena váha 1. A tím, jak se částice toulá po daném systému, tak interaguje s prostředím až do místa, kde by teoreticky mělo dojít k její absorbci. Nicméně, nedojde k úplné absorbci, pouze se dojde k započítání do skórování a dále se jí srazí váha na nižší hodnotu. Nedojde tedy k zániku, a jedna částice může pokračovat v toulání se po systému. Pouze má nižší váhu, proto narazí-li na další absorbci, opět dojde k započtení do skórování, ale s nižší váhou, jelikož v tomto druhém místě by už částice neměla teoreticky existovat. Dojde-li po několika srážkách ke snížení váhy pod jistou hladinu, až tehdy dojde k celkovému zániku dané částice.\\

Tímto způsobem se dají zpřesnit výsledky za pomoci menšího počtu částic. Nicméně, jde o komplikovanější metodu a je potřeba znát širší matematický základ. V prvním případě je důležité určit, jakým zpsůobem se změní váha dané částice. Tato technika se nazývá \textbf{redukce variance} (redukce variance, jelikož zpřesňují výsledek a snižuji rozptyl, tedy varianci).

\subsection{Redukce variance}

První možností při sražení váhy je za pomoci teoretické absorbce. To se určí jednoduše dle vztahu:

$$ w = w' \left ( 1-\dfrac{\Sigma_\text{a}}{\Sigma_\text{t}} \right ). $$

Dále je třeba určit, při jaké energii dojde ke skutečnému zániku částice, jelikož trekování částice s příliš malou váhou je naopak kontraproduktivní. To se určuje dle tzv. Ruské rulety, jejíž nejjednodušší zápis je ve tvaru:

$$ P = 1-\dfrac{w}{w_0}, $$

kde $w_0$ značí váhu na začátku historie a $P$ značí pravděpodobnost, při které dojde k zániku.

Dále je možné podobnou metodu aplikovat na vznik částic (z hlediska neutronů štěpení (n,xn) reakce). V takovém případě se nová váha určí jako:

$$ w = w' \nu. $$

Tohle jsou jenom základní metody, ve skutečnosti je nutné aplikovat více metod redukce variance. Těmi hlavními jsou:

\begin{itemize}
  \item cut-off -- zánik při poklesu pod jistou hladinu (energetické, váhové, časové apod.),
  \item splitting -- rozdělení na 2 částice, každá s jinou váhou (geometrické, energetické, časové),
  \item ruská ruleta -- naopak zánik dané částice, kombinace se splittingem (velká váha splitting, malá váha ruská ruleta),
  \item biassing -- to je to zmiňované sražení váhy (absorbcí).
\end{itemize}

Nicméně nejdůležitější je pochopitelně kombinace všeho.

% \include{chapter_08}
% \section{Výpočty vyhoření jaderného paliva a příprava homogenizovaných dat pro celozónové výpočty}

\subsection{Transportní (deterministický) výpočet}

Ve zkratce: 

\begin{enumerate}
  \item[1.] Nejprve si vymyslím, jakou knihovnu dat použiji a pro tu šáhnu (např. \textbf{ENDF/B}. Jde o bodové knihovny a každá má různý počet bodů pro různé materiály).
  \item[2.] Tato nezpracovaná data je potřeba zpracovat (zgrupovat, upravit na teplotu apod.) vhodným nástrojem (např. \textbf{NJOY}), čímž vzniká pracovní knihovna obsahující stovky až tisíce grup.
  \item[3.] Následuje \textbf{Lattice výpočet} pomocí nějakého mikrokódu (např. \textbf{Newt/TRITON}, \textbf{Helios}), často v nekonečné mříži (pro tepelné systémy to stačí) pomocí transportní teorie. Tento model už by měl zohledňovat samostínění. Homogenizovat je možné i stochasticky bez aplikace transportní teorie (např. \textbf{Serpent}).
  \item[4.] Tím dochází k homogenizaci a k tvorbě homogenizovaných dat (difúzní koeficient, makroskopické průřezy apod.) pro daný palivový soubor. Zároveň se slučují grupy do menšího počtu (tepelný systém stačí 2, rychlé 4, lépe 8, záleží na složení. Neplechu dělá větší množství Pu, kde je potřeba více grup).
  \item[5.] Homogenizované veličiny putují do celozónového výpočtu makrokódem (např. \textbf{PARCS}, \textbf{Andrea}), který častou pouze s pomocí difúzního přiblížení určí hledané parametry (kritičnost, energetická distribuce, vyhoření apod.).
\end{enumerate}

\subsubsection{Homogenizace}

Pro deterministické kódy se postupuje podobně, jako při grupování mikroskopických průřezů, akorát že teď se grupují makroskopické průřezy vážené hustotou toku a nově i objemem. Opět se musí zachovat reakční rychlost:

\begin{equation}
  \boxed{
    \Sigma_G = \dfrac{\sum_{h \in G} \sum_i V_i \Sigma_{i,h} \Phi_{i,h}}{\sum_{h \in G} \sum_i V_i \Phi_{i,h}}.}
\end{equation}

Problém je, že takto homogenizovaná data jde použít pouze na původně řešený problém, nepaltí tedy obecně. Při změně geometrie se musí homogenizovat znovu. Pro následující nodální výpočet nám jde o:

\begin{itemize}
  \item $\Sigma_\text{tr}$ -- převrácená hodnota představuje dráhu, kterou neutron urazí po nekonečném množství srážek. Potřebujeme znát pro výpočet difúzního koeficientu, který se používá v difúzním přiblížení. Pro jeho určení se vychází z In-scatter nebo Out-scatter aproximace (nebo obojího), což je rozdíl totálního průřezu a sumy rozptylových průřezů z rychlé grupy.
  \item $\Sigma_\text{a}$ -- představuje zánik neutronů.
  \item $\nu \Sigma_\text{f}$ -- představuje produkci neutronů.
  \item $\kappa \Sigma_\text{f}$ -- představuje energii ze štěpení.
  \item $\Sigma_\text{s, g - g'}$ -- vyjadřuje přechod mezi grupami. 
  \item relativní výkony jednotlivých proutků.
\end{itemize}

Pro zachování spojitosti hustoty proudu mezi soubory se využívají tzv. ADF (Assembly Discontinuity Factors).\\

Homogenizace ve štěpném prostředí (speciálně pro tepelné reaktory, rychlé si myslím že jsou obtížnější) probíhá v nekonečné mříži za použití zrcadlových hraničních podmínek. To má v sobě některá úskalí:

\begin{itemize}
  \item Neuvažuje se únik, který tam ve skutečnosti je. Ten je možné opravit pomocí B1 aproximace, což je oprava přes hranice souboru na základě kritičnosti. Jednoduše řečeno, k totální grupové reakční rychlosti se přičte/odečte B-násobek grupové hustoty proudu tak, aby došlo k dosažení kritického stavu, čímž se získají zbylé homogenizované průřezy.
  \item Neuvažuje se prostorový gradient hustoty toku (v celozónovém výpočtu není hustota toku ve všech souborech středově souměrná). Oprava pomocí rehomogenizace, která kombinuje hustotu toku v nekonečné mříži a hustotu toku z celozónového výpočtu, která se získala po dosazení původních dat. Je důležité hlavně v blízkosti silných absorbátorů, kde je gradient hustoty toku významný.
  \item Spektrum (teď nemyslím prostorové, ale energetické) neodpovídá skutečnosti.
\end{itemize}

Homogenizace v neštěpném prostředí je obtížnější, protože tady už tuplem neznáme spektrum neutronů. Jsou zde jiné hraniční podmínky (vacuum v okolí reflektoru, reflective/periodic v okolí souborů), nepoužívá se B1 aproximace. Pro anizotropní rozptyl na vodíku se zavádí speciální opravná funkce, která se dá získat výpočtem v 1D prostředí pomocí In-scatter aproximace.


\subsubsection{Parametrizace}

Homogenizace se provádí pro různé stavy řešeného systému, tzv. odskoky. Celozónový výpočet totiž potřebuje data, která jsou závislá na provozu reaktoru (změní-li se teplota paliva, jak se změní homogenizovaná data). Odskoky se provádí např. na teplotě (moderátoru i paliva), vyhoření, hustotě (moderátoru) a koncentraci absorbátoru.\\

Odskoky fungují tak, že se změní zkoumaný parametr (stačí jednou nahoru a podruhé dolu), systém se zhomogenizuje a výsledná data se proloží lineárním (či jiným) fitem. Tak se získá závislost homogenizované veličiny na měnícím se parametru. Jde to i složitě, mohou se např. měnit 2 parametry najednou (což dle Pavla dává mnohem lepší výsledky), prokládání fitem se může lišit apod.\\

Mzi odskoky v klasické provozní fyzice patří:

\begin{itemize}
  \item teplota a hustota moderátoru $\dfrac{\partial \Sigma}{\partial T_\text{M}}$,
  \item teplota paliva $\dfrac{\partial \Sigma}{\partial \sqrt{T_\text{P}}}$,
  \item koncentrace kyseliny borité $\dfrac{\partial \Sigma}{\partial C_\text{B}}$,
  \item regulačních tyčí $\alpha$.
\end{itemize}

Takto získám původní provozní sadu průřezů, parametrizované odskoky a pro nové průřezy pak platí:

\begin{equation}
  \boxed{
    \Sigma (T_\text{M}, T_\text{P}, C_\text{B}, \text{CR}) = \Sigma + \dfrac{\partial \Sigma}{\partial T_\text{M}} \Delta T_\text{M} + \dfrac{\partial \Sigma}{\partial \sqrt{T_\text{P}}} \Delta \sqrt{T_\text{P}} + \dfrac{\partial \Sigma}{\partial C_\text{B}} \Delta C_\text{B} + \alpha \Delta \Sigma_\text{CR}.}
\end{equation}

\subsubsection{Celozónový výpočet}

Po takto zhomogenizovaných a zparametrizovaných souborech následuje celozónový výpočet. Ten probíhá buď pomocí zjednodušeného difúzního přístupu, nebo pomocí Simplified P3 metody transportní rovnice (SP3).\\

Obecně jsou 2 přístupy:

\begin{itemize}
  \item Diferenční přístup -- derivace se nahradí diferencemi a postupuje se pomocí sítí. Je to metodicky snažší, ale zdlouhavé. Objemové elementy jsou mezi sebou vázány přes hustotu toku (rovnost na hranici) a systém vede na soustavu rovnic, která je řešitelná triagonální maticí se 2 neznámými ($\Phi$ a $k_\text{ef}$). Nejprve se provede počáteční odhad a pak se iteruje, dokud se nedokonverguje k nějaké hodnotě $\Phi$ (vnitřní iterace), pomocí které se získá nový $k_\text{ef}$ (podělí se zdroj neutronů z iterace 0 a 1, vnější iterace). Vnější iterace končí po dosažení nějakého kritéria.
  \item Nodální přístup -- hustota toku se rozvine do bazických funkcí, a pak je možné redukovat objemové elementy až na úroveň celého souboru (nódy). Rychlejší, ale fakt složité.
\end{itemize}

Celozónové kódy musí být rychlé a jednoduché. Dále se nabízí coupling s termohydraulickým kódem (RELAP, TRACE), ale to je už fakt hardcore, protože nikdo pořádně neví, jak k tomu přistupovat.

\subsection{Výpočty vyhoření}

\end{document}
