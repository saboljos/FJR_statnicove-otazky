\section[Difúzní teorie]{Odvození a využití difuzní teorie v reaktorové fyzice}

\textbf{Difúzní rovnice} představuje základní rovnici stanovující prostorové rozložení neutronů v látce. Přesný popis neutronů je příliš komplikovaný, proto je vhodné volit určitá zjednodušení. Pro nejobecnější popis je nutné použít Boltzmannovu transportní rovnici, nicméně s určitými zanedbáními postačuje pouze difúuzní rovnice, případně difúzní rovnice s transportními korekčními faktory.

\subsection{Fickův zákon}

Difúzní rovnice je založena na \textbf{Fickově zákonu}, který je možné znát např. z chemie (šíření látky proti směru gradientu koncentrace):

\begin{equation}
    \boxed{ \textbf{J} = - D \text{grad} \Phi = -D \nabla \phi, }
\end{equation}

kde $\textbf{J}$ značí hustotu proudu neutronů (vektor), $D$ značí difúzni koeficient a $\Phi$ značí hustotu toku neutronů. To samé z části platí i pro neutrony. Pokud je hustota toku neutronů v jedné oblasti vyšší, neutrony putují opačným směrem. 

To a jak jsou neutrony ochotny se v dané látce šířít pak popisuje \textbf{difúzní koeficient} $D$. Difúzní rovnice předpokládá, že je difúzní koeficient funkcí materiálu, nikoliv polohy (izotropní rozptyl). V případě slabého anizotropního rozptylu je možné difúzní koeficient zavést za pomoci \textbf{střední volné dráhy pro transport} $\lambda_\text{tr}$ jako:

$$ D = \dfrac{\lambda_\text{tr}}{3}, $$

$$ \lambda_\text{tr} = \dfrac{1}{\Sigma_\text{tr}} = \dfrac{1}{\Sigma_\text{s}(1-\bar{\mu})}, $$

kde střední volná dráha pro transport udává průměrnou vzdálenost, kterou neutron urazí v původním směru po nekonečném množství rozptylových srážek. Pro izotropní rozptyl v těžišťové soustavě pak navíc platí:

$$ \bar{\mu} = \dfrac{2}{3A}. $$

$\bar{\mu}$ zde označuje cosinus úhlu rozptylu.

Obecně Fickův zákon platí všude, až na:

\begin{itemize}
    \item silně absorbční prostředí (absorbátory),
    \item vzdálenost menší než 3 střední volné dráhy pro transport od zdroje, nebo od hranic difúzního prostředí,
    \item silně anizotropní rozptyl.
\end{itemize}

Tyto podmínky jsou částečně všude, proto je potřeba k nim přistupovat opatrně, případně hledat jiná přiblížení.

\subsection{Odvození}

Odvození vychází z rovnice kontinuity. Pokud si představíme element objemu $V$, pak musí platit:

\begin{equation*}
    \boxed{
    \begin{pmatrix} \text{rychlost změny} \\ \text{počtu neutronů} \end{pmatrix} = \begin{pmatrix} \text{rychlost produkce} \\ \text{neutronů} \end{pmatrix} - \begin{pmatrix} \text{rychlost absorbce} \\ \text{neutronů} \end{pmatrix} - \begin{pmatrix} \text{rychlost úniku} \\ \text{(difuze) neutronů} \end{pmatrix}.
    }
\end{equation*}

Pokud $n(\textbf{r})$ označuje hustotu neutronů, pak \textbf{rychlost změny počtu neutronů} lze určit jako:

$$ \dfrac{\text{d}}{\text{d}t} \int_V n(\textbf{r}) \: \text{d} V, $$

přičemž derivaci je možné z nějaké chytré matematické věty šoupnout do integrálu, tedy:

$$ \boxed{\int_V \dfrac{\partial n}{\partial t} \: \text{d} V.} $$

\textbf{Rychlost produkce}, resp. \textbf{rychlost absorbce} je možné zapsat za pomoci reakčních rychlostí:

$$ \boxed{\int_V s(\textbf{r}) + \nu \Sigma_\text{f} \Phi(\textbf{r}) \: \text{d} V,} $$

$$ \boxed{\int_V \Sigma_\text{a} \Phi(\textbf{r}) \: \text{d} V.} $$

Rychlost úniku z elementu $V$ je pak dána difúzí a aplikací Fickova zákonu přes plochu $S$:

$$ \oint_S \textbf{J} \cdot \text{d} \textbf{S}, $$

což po aplikaci další chytré matematické věty pojmenované po nějakém slavném matematikovi přejde na:

$$ \boxed{\int_V \text{div} \textbf{J} \: \text{d} V.} $$

Teď už se to jenom poskládá dohromady, odstraní integrál (předpoklad, že to platí všude, tudíž se z integrální formy přejde do diferenciální), z divergence gradientu se udělá Laplacián (předpoklad, že $D$ není funkcí prostoru, a tudíž půjde vytkntout) a ještě se od hustoty neutronů přejde k hustotě toku neutronů, čímž dostaneme:

\begin{equation}
    \boxed{
        \dfrac{\partial n}{\partial t} = \dfrac{1}{v} \dfrac{\partial \Phi}{\partial t} = D \Delta \Phi - \Sigma_\text{a} \Phi + s + \nu \Sigma_\text{f} \Phi.
    }
\end{equation}

Ještě je možný jeden zápis, kdy difúuzní rovnici podělím difúzním koeficientem, čímž se do rovnice dostane difúzní délka $L$, resp. difúzní plocha $L^2$:

$$ L^2 = \dfrac{D}{\Sigma_\text{a}}, $$

\begin{equation}
    \boxed{
        \dfrac{1}{D v} \dfrac{\partial \Phi}{\partial t} = \Delta \Phi - \dfrac{1}{L^2} \Phi + \dfrac{s}{D} + \dfrac{\nu \Sigma_\text{f}}{D} \Phi.
    }
\end{equation}

\textbf{Difúzní plocha} popisuje látku vzhledem k tomu, kolik je neutron schopný procestovat v prostředí před tím, než bude absorbován. Pokud se bude uvažovat pravděpodobnost absorbce, která se zintegruje přes trasu a nalezne se střední hodnota tak vyjde, že $L^2$ představuje 1/6 střední hodnoty čtverce přímé zvdálenosti mezi místem vzniku a místem absorbce.

Hodnoty $L$ a $D$ se liší od materiálu, pro základní moderátory platí:

\begin{table}[h!]
    \centering
    \begin{tabular}{lccc}
    \toprule
    \textbf{Moderátor} & \textbf{hustota (g/cm$^3$)} & $D$ \textbf{(cm)} & $L$ \textbf{(cm)} \\ \midrule
    H\textsubscript{2}O & 1,00 & 0,13 & 2,62 \\
    D\textsubscript{2}O & 1,10 & 0,76 & 147 \\
    Be & 1,85 & 0,45 & 22 \\
    Grafit & 1,60 & 0,87 & 61 \\ \bottomrule
    \end{tabular}
    \caption{Tabulka vlastností moderátorů.}
\end{table}
    

\subsection{Řešení}

Jedná se o diferenciální rovnici, tudíž pro její řešení je nutné stanovit okrajové podmínky. Mezi ty patří:

\begin{itemize}
    \item Konečnost a nezápornost hustoty toku neutronů $0 \leq \Phi < \infty $.
    \item Rovnost hustoty toku a proudu na rozhraní $\Phi_A = \Phi_B$ \& $\textbf{J}_A = \textbf{J}_B$.
    \item Okrajová podmínka na vnějším rozhraní.
    \item Zdrojové podmínky.
\end{itemize}

Podmínku na vnějším rozhraní je možné zavést za pomoci \textbf{extrapolované délky} $d$, jelikož bylo zjištěno, že pokud hustota toku klesne na nulu ve vzdálenosti $d$ od rozhraní, pak se spočtené rozložení blíží realitě. Extrapolovanou délku je možné stanovit jako:

$$ d = 0,71 \lambda_\text{tr} = 2,13 D. $$

Zdrojové podmínky se týkají geometrie \textbf{zdroje} $s$. Jelikož difúzní rovnice neplatí v místě zdroje, je nezbytné tyto oblasti řešit samostatně. Postupuje se tak, že se difúzka řeší mimo zdroj, přičemž zdroj samotný se obklopí plochou, kterou prochází všechny emitované neutrony o vydatnosti $S$. Limitní podmínku je pak možné stanovit dle geometrie jako:

\begin{itemize}
    \item $ \lim_{x \to 0} 2 \: \textbf{J}(x) = S $ pro rovinný zdroj (2 plochy),
    \item $ \lim_{r \to 0} 4 \pi r^2 \: \textbf{J}(r) = S $ pro bodový zdroj (plocha sféry),
    \item $ \lim_{r \to 0} 2 \pi r \: \textbf{J}(r) = S $ pro přímkový zdroj (obvod kružnice).
\end{itemize}

Difúzka je jednoduše řešitelná i analyticky, kor stacionární případ. Stačí nalézt vhodného Laplaciána dle geometrie a přepsat rovnici do tvaru:

$$ \Delta \Phi \pm A^2 = 0 $$

Výsledné řešení je pak závislé na znamínku. Pokud převládá absorbce nad produkcí (ZAF1, difúze v prostředí $ - A^2$), vede řešení na hyperbolické funkce a Modifikované Besselovy funkce typu $I_n$ a $K_n$. Pokud převládá produkce nad absorbcí (ZAF2, homogenní reaktor $ + A^2$), vede řešení na goniometrické funkce a Besselovy funkce typu $J_n$ a $Y_n$.

\subsection{Grupová difúzní rovnice}

Difúzní rovnice je možné zapsat i s energetickou závislotí za pomoci rozgrupování. Poté se pro každou grupu řeší vlastní difúzka, přičemž přechod mezi grupami se zapisuje pomocí \textbf{rozptylového grupového průžezu} $\Sigma_{m \to n}$, tedy pravděpodobnost přechodu z grupy $m$ do grupy $n$. Grupa s nejvyšší energií se označuje indexem $1$ a s klesající energií roste číslo grupy. Skupinová difúzní rovnice pro grupu $g$ pak vypadá (stacionární případ, kde zdroj ze štěpení je zahrnut v koeficientu $s$):

\begin{equation}
    \boxed{
        D_g \Delta \Phi_g - \Sigma_\text{a,g} \Phi_g - \sum^N_{h=g+1} \Sigma_{g \to h} \Phi_g + \sum^{g-1}_{h=1} \Sigma_{h \to g} \Phi_h = -s_g.
    }
\end{equation}

Zde předpokládáme pouze downscattering, tedy přesun z vyšší grup do nižší grupy (resp. z nižšího pořadového čísla do vyššího). To obecně platí pro vysoké energie (neutrony vznikají s energií v řádech MeVů a postupně termalizují, tedy pouze ztrácejí energii rozptylovými srážkami), na úrovní tepelné energie se může objevovat i upscattering a celkový popis je složitější.

\subsection{2G přiblížení \& grupování}

Obecně je nutné uvažovat alespoň 2 grupy, tepelné a rychlé neutrony s hranicí na 0,625 eV. Ukazuje se, že hustotu toku tepelných neutronů je možné popsat za pomoci \textbf{Maxwellovy distribuce}. Pak je možné účinné průřezy vystředovat a uvažovat celou tepelnou oblast jako jednu grupu se středovanými průřezy. 

Maxwellova distribuce je dána:

$$ n(E) = \dfrac{2 \pi n}{(\pi k T)^{3/2}} \sqrt{E} \: \text{exp} \left(- \dfrac{E}{kT}\right), $$

rychlost je definuje:

$$ v(E) = \sqrt{\dfrac{2E}{m}}, $$

a pak je možné určit hustotu toku tepelných neutronů $\phi_T$ jako:

$$ \Phi_T = \int_T \Phi(E) \: \text{d}E = \int_T n(E) v(E) \: \text{d}E = \dfrac{2n}{\sqrt{\pi}} \sqrt{\dfrac{2kT}{m}} $$

Tím se určí střední hustota toku tepelných neutronů, která se pak aplikuje na středované průřezy:

$$ \bar{\Sigma} = \dfrac{1}{\Phi_T} \int_T \Sigma(E) \Phi(E) \: \text{d}E.$$

Pokud navíc platí oblast $1/v$ (a pokud ne, přidá se korekční faktor $g$), tak je možné nahradit:

$$ \Sigma(E) \Phi(E) = \Sigma(E_0) \Phi_0, $$

kde index 0 odpovídá energii 0,0253 eV, tedy 293 K, což se nalezne v tabulkách. Celkové výrazy pak budou:

$$ \boxed{\bar{\Sigma} = \dfrac{\sqrt{\pi}}{2} g(T) \Sigma(E_0) \sqrt{\dfrac{T_0}{T}},} $$

což platí pro všechny hodnoty (průřezy, energie, difúzní koeficienty apod.).

Nyní je už možné zkompletovat 2G sadu difúzních rovnic:

\begin{equation}
    \boxed{
        \Delta \Phi_T - \dfrac{1}{L_T^2} \Phi_T = -\dfrac{\Sigma_1 \Phi_1}{\bar{D}},
    }
\end{equation}

\begin{equation}
    \boxed{
        \Delta \Phi_1 - \dfrac{1}{\tau} \Phi_1 = -\dfrac{s_1}{D_1},
    }
\end{equation}

kde $\tau$ označuje Fermiho stáří neutronů (1/6 čtverce přímé vzdálenosti mezi vznikem a přechodem do tepelné grupy, příp. absorbce), a kde zdrojem tepelných neutronů je scattering z rychlé grupy, a zdroje v rychlé grupě jsou označeny jako $s_1$ (např. ze štěpení).

\subsection{Využití}

Difúzka se využívá jako první přiblížení, pokud nás zajímá transport neutronů, rozložení neutronového pole, kritičnost reaktoru, efekt moderátoru a reflektoru apod. Využívá se i v praxi, jelikož její řešení je velmi jednoduché. Např. pokud řešíme celozónový výpočet, tak většina všech full-core nástrojů využívají právě difuzní řešení (Andrea, PARCS apod.), které se řeší numericky (nodálně, sítěmi apod.). Z lattice kódu se vytáhne sada účinných průřezů a v full-core kódu se tyhle průřezy nastrkají do difúzní rovnice.

Mezi rovnice odvozené v rámci ZAF patří:

\subsubsection{Modifikovaná 2G kritická rovnice}

Vychází z předpokladu, že máme homogenní systém popsaný za pomoci 2G rovnice. Neutrony vznikají ze štěpení jako rychlé neutrony, přecházejí do tepelné oblasti kde inicializují štěpení. Celá úloha je pak zestaciarizována za pomoci koeficientu násobení. Ke štěpení rychlými neutrony nedochází vůbec.

$$ k_\text{ef} = \dfrac{k_\infty}{1 + B^2 \cdot M^2}, $$

kde $k_\infty$ představuje koeficient násobení pro systém bez úniku, $B^2$ představuje geometrický faktor (dáno geometrií) a $M^2$ představuje migrační plochu (dáno materiálem).

Na rovnici je možné pohlížet i jako na pravděpodobnost, že neutron neunikne $P_\text{NL}$. Pokud se navíc migrační plocha přepíše jako suma difúzní plochy (charakterizuje transport tepelných neutronů, tedy proces difuze) a Fermiho stáří (charakterizuje transport rychlých neutronů, tedy proces termalizace), pak platí:

$$ P_\text{NL} = \dfrac{k_\text{ef}}{k_\infty} = \dfrac{1}{1 + B^2 \cdot (L^2 + \tau)} \approx \dfrac{1}{1 + B^2 \cdot L^2} \cdot \dfrac{1}{1 + B^2 \cdot \tau} = P_\text{TNL} \cdot P_\text{FNL}. $$

Rovnici je možné i zobecnit pro konkrétnější případy, více grup apod.