\section[Vyhořívání]{Výpočty vyhoření jaderného paliva, formulace úlohy a metoda prediktor-korektor}

UPRAVIT!!

\subsection{Výpočty vyhoření}

Vyhoření se počítá pomocí Batemanových rovnic:

\begin{equation}
  \dfrac{\text{d}n_i}{\text{d}t} = -\sigma_i \phi n_i + \sum_{j \to i} \gamma_i \sigma_j \phi n_j - \lambda_i n_i + \sum_{j \to i} \gamma_i \lambda_j n_j,
\end{equation}

nicméně aby šly dopočíst, je třeba znát hodnotu $\phi$, která se časem mění. Proto se postupuje pomocí tzv. metody \textbf{prediktor-korektor}, která nejprve odhadne $\phi$ na začátku cyklu, dopočte na konec cyklu (prediktor), přepočte novou $\phi$ z konce. Z těchto dvou hodnot poté stanoví průměr (korektor) a dopočte nové složení na konci cyklu.

Proto je důležité vhodně stanovit délku kroku. Například u tepelných systémů je třeba počáteční kroky volit velmi krátké, jelikož dochází k ustalování koncentrace xenonu, což má za následek velký nárůst $\phi$. U rychlých systémů to tak kritické není.

Nicméně, tato metoda je pouze na odhad hustoty toku. Stále máme kombinaci LDR o hodně členech (např. ENDF/B-VIII.0 1600 nuklidů). Vypíšu 2 metody řešení:

\subsubsection{MATREX metoda}

Maticovou formu Batemanových rovnic:

\begin{equation}
  \dfrac{\text{d}\vec{N}}{\text{d}t} = \textbf{A} \vec{N}(t)
\end{equation}

řeším ve formě exponenciály:

\begin{equation}
  \vec{N}(t) = \text{exp} (\textbf{A}t) \vec{N}(0),
\end{equation}

$$ \text{exp} (\textbf{A}t) = \sum_{k=0}^\infty \dfrac{(\textbf{A}t)^k}{k!}. $$

\subsubsection{CRAM metoda}

Neboli aproximace pomocí Chebyshevových racionálních funkcí. Obyčejně se využívá CRAM metoda 16. řádu, která má stejnou přesnost, jenom je rychlejší než MATREX metoda.

Funguje na stejném základu, pouze rozkládá exponenciální matici do Chebyshevových racionálních funkcí:

$$ \vec{N}(t) = \left [ \sum_{k=0}^\infty \dfrac{(\textbf{A}t)^k}{k!} \right ] \vec{N}(0) \approx \left [ \alpha_{0,K} \textbf{I} + 2 \text{Re} \left [ \sum_{j=1}^{K/2} \left ( -\dfrac{\alpha_{j,K}}{\theta_{j,K}} \right ) \left ( \textbf{I} + \textbf{A} \left ( -\dfrac{t}{\theta_{j,K}} \right ) \right ) ^{-1} \right ] \right ] \vec{N}(0), $$

kde $K$ představuje řád a $\alpha_{i,j}$ a $\theta_{i,j}$ jsou příslušné CRAMerovy koeficienty k dohledání v literatuře.